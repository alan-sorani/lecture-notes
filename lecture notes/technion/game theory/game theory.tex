\documentclass[a4paper,10pt,twoside,openany]{book}

\usepackage[lang=hebrew]{maths}
\usepackage{hebrewdoc}
\usepackage{stylish}
\usepackage{lipsum}

\title{סיכומי הרצאות בתורת המשחקים \\ \large{חורף 2018, הטכניון}}
\author{הרצאותיו של רון הולצמן \\ \large סוכמו על ידי אלעד צורני}
\date{\today}

\begin{document}
\frontmatter
\frontpage{chess}{0.8\textwidth}
\tableofcontents
\countlectures
\newpage

\chapter*{הקדמה}
\addcontentsline{toc}{chapter}{הקדמה} \markboth{הקדמה}{}

\section*{הבהרה}
\addcontentsline{toc}{section}{הבהרה} %\markboth{Technicalities}{}

סיכומי הרצאות אלו אינם רשמיים ולכן אין
\emph{כל הבטחה}
כי החומר המוקלד הינו בהתאמה כלשהי עם דרישות הקורס, או שהינו חסר טעויות.
\\
להיפך, ודאי ישנן טעויות בסיכום! אעריך אם הערות ותיקונים ישלחו אלי בכתובת דוא"ל
\textenglish{\href{mailto:tzorani.elad@gmail.com}{tzorani.elad@gmail.com}}.\\
אלעד צורני.

\section*{ספרות מומלצת.}
\addcontentsline{toc}{section}{ספרות מומלצת} %\markboth{Course Literature}{}

הספרות המומלצת עבור הקורס הינה כדלהלן.

\begin{english}
\begin{description}
\item \stress{G. Owen}: Game Theory (Academic Press)
\end{description}
\end{english}

\section*{פרטים טכניים}
הקורס מועבר על ידי פרופ' רון הולצמן בטכניון.
שעת הקבלה בימי א' בין השעות 15:30 ו־16:30.

\section*{על תורת המשחקים}

תורת המשחקים פותחה עם כתיבת הספר
\emph{תורת המשחקים והתנהגות כלכלית}
מאת המתמטיקאי ג'ון פון־נוימן והכלכלן אוסקר מורגנשטרן.
עם כן, תורת המשחקים פותחה כתחום מתמטי במחשבה על שימושיה לכלכלה, וכיום הינה כלי מרכזי בחקר כלכלה.
אכן הוענקו פרסי נובל בכלכלה על חקר תורת המספרים, בין השאר עבור המתמטיקאי ג'ון נאש שתרם את עיקרון שיווי המשקל.
במהלך השנים תורת המשחקים חדרה לתחומים נוספים, בינם תורת האבולוציה, ביולוגיה, וחקר מערכות במדעי המחשב.

החלוקה המרכזית בתורת המשחקים היא חלוקה לתחומים של משחקים שיתופיים ושאינם שיתופיים.
בקורס זה נעסוק בשני התחומים. למרות שנראה כי אין קשר בינם, נגלה כי דווקא קיים קשר שכזה.
בצורה כללית ביותר, במשחקים שאינם שיתופיים ההנחה הבסיסית היא שכל אחד מהשחקנים פועל להשגת מטרותיו שלו, והינו אוטונומי לקבלת החלטות לקידום מטרותיו.
במשחקים שיתופיים, השחקנים יוצרים קואליציות בינם ופועלים במשותף כדי לקדם את מטרות אותה קואליציה.

\mainmatter

\part{משחקים שאינם שיתופיים}

\chapter{משחקים דמויי־שחמט}

\section{מאפיינים}

המאפיינים של משחקים דמויי־שחמט הם:
\begin{itemize}
\item יש שני שחקנים
שנקרא להם "לבן" ו־"שחור".
\item שני השחקנים משחקים לסירוגין.
\item בכל שלב שבו שחקן צריך לשחק, הוא יודע את כל מהלך המשחק עד אותו שלב.
\item מהלך המשחק נקבע באופן מלא ע"י החלטות השחקנים: אין צעדי גורל.
\item תוצאת המשחק היא אחת מהבאות.
\begin{itemize}
\item נצחון ללבן.
\item נצחון לשחור.
\item תיקו.
\end{itemize}
\end{itemize}

נתאר משחק דמוי־שחמט בעזרת
stress{עץ משחק}.
השלב ההתחלתי הינו שורש העץ
$r$.
נציין ליד כל קודקוד את זהות השחקן הבוחר את המסע הבא,
$b$
עבור שחור או
$w$
עבור לבן.
הצלעות יתאימו למסעים האפשריים עבורו.
המשחק יכול להסתיים במספר סופי של מסעים, ואם הדבר קורה נקבעת תוצאה (נצחון לאחד הצדדים, או תיקו). לכן, אם הגענו לקודקוד קצה ניתן לרשום עליו את תוצאת המשחק. נסמן
$B$
עבור ניצחון של השחור,
$W$
עבור ניצחון של הלבן, ו־%
$D$
עבור תיקו.

\section{הגדרות}

\begin{definition}
\stress{משחק דמוי־שחמט}
מתואר על ידי עץ משחק ובו המרכיבים הבאים:
\begin{itemize}
\item עץ
$T = \prs{V,E}$ –
כלומר גרף עם קבוצה
$V$
(תיתכן אינסופית) של קודקודים, וקבוצה
$E$
של צלעות, שהוא קשיר וחסר־מעגלים.
\item קודקוד $r$
ב־%
$V$
הנקרא
\stress{שורש}.
אנו חושבים על כל צלע כמכוונת בכיוון המתרחק מן השורש.
\item קבוצת הצלעות היוצאות מקודקוד
$v$
נקראת
\stress{קבוצת המסעים האפשריים ב־%
$v$}.
\item
קודקוד שאין לו עוקבים נקרא
\stress{קודקוד קצה.}
\item
קבוצת קודקודי הקצה תסומן ע"י
$V_{\text{end}}$.
\item
קבוצת הקודקודים שאינם קודקודי־קצה מחולקת לשתי תתי־קבוצות
$V_{\text{white}}, V_{\text{black}}$,
כלומר
\[V \setminus V_{\text{end}} = V_{\text{white}} \cup V_{\text{black}}\]
כאשר
$V_{\text{white}}$
היא קבוצת הקודקודים במרחק זוגי מהשורש ו־%
$V_{\text{black}}$
היא קבוצת הקודקודים במרחק אי־זוגי מהשורש, או להיפף.
קודקודי
$V_{\text{white}}$
נקראים קודקודי ההחלטה של לבן, קודקודי
$V_{\text{black}}$
נקראים קודקודי ההחלטה של שחור.
\item \stress{תחרות} (\textenglish{play})
היא מסלול בעץ המשחק המתחיל בשורש ומקיים אחד משני הבאים.
\begin{enumerate}[label = \alph*.]
\item מסתיים בקודקוד קצה
\item נמשך לבלי סוף
\end{enumerate}
\item קבוצת כל התחרויות תסומן ע"י
$P$.
\item כחלק מתיאור המשחק, נתונה חלוקה
$P = W \amalg B \amalg D$
(זהו סימון לאיחוד זר)
המחלקת את התחרויות לכאלה שהמהוות נצחון ללבן, ניצחון לשחור, או תיקו.
\end{itemize}
\end{definition}
\begin{remark}
חייבת להיות תוצאה למשחק, אך היא יכולה להתקבל לארך אין־סוף מסעים.
לדוגמה, אם שני שחקנים בוחרים ספרות למספר עשרוני בין
$0$
ו־%
$1$.
הלבן מנצח אם המספר הינו רציונלי, והשחור אחרת.
\end{remark}
\begin{definition}
במשחק דמוי שחמט,
\stress{תכסיס} (\textenglish{strategy})
של שחקן הוא כלל האומר לו באיזה מסע לבחור בכל אחד מקודקודי ההחלטה שלו.
כלומר, תכסיס של לבן, זוהי פונקציה
$\sigma \colon V_{\text{white}} \to E$
כך שלכל
$v \in V_{\text{white}}$,
מתקיים כי
$\sigma\prs{v}$
מסע אפשרי ב־%
$v$.
באופן דומה, תכסיס של שחור זוהי פונקציה
$\tau \colon V_{\text{black}} \to E$
כך שלכל
$v \in V_{\text{black}}$,
מתקיים כי
$\tau\prs{v}$
הוא מסע אפשרי ב־%
$v$.
\end{definition}
\backmatter
\end{document}

