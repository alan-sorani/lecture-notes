\documentclass[a4paper,10pt,twoside,openany]{book}

\usepackage[lang=hebrew]{maths}
\usepackage{hebrewdoc}
\usepackage{stylish}
\usepackage{lipsum}

\title{סיכומי הרצאות בתורת המשחקים \\ \large{חורף 2018, הטכניון}}
\author{הרצאותיו של רון הולצמן \\ \large סוכמו על ידי מתן שרגר ואלעד צורני}
\date{\today}

\DeclareMathOperator{\depth}{depth}

\begin{document}
\frontmatter
\frontpage{chess}{0.8\textwidth}
\tableofcontents
\countlectures
\newpage

\chapter*{הקדמה}
\addcontentsline{toc}{chapter}{הקדמה} \markboth{הקדמה}{}

\section*{הבהרה}
\addcontentsline{toc}{section}{הבהרה} %\markboth{Technicalities}{}

סיכומי הרצאות אלו אינם רשמיים ולכן אין
\emph{כל הבטחה}
כי החומר המוקלד הינו בהתאמה כלשהי עם דרישות הקורס, או שהינו חסר טעויות.
\\
להיפך, ודאי ישנן טעויות בסיכום! אעריך אם הערות ותיקונים ישלחו אלי בכתובת דוא"ל
\textenglish{\href{mailto:tzorani.elad@gmail.com}{tzorani.elad@gmail.com}}.\\
אלעד צורני.

\section*{ספרות מומלצת.}
\addcontentsline{toc}{section}{ספרות מומלצת} %\markboth{Course Literature}{}

הספרות המומלצת עבור הקורס הינה כדלהלן.

\begin{english}
\begin{description}
\item \stress{G. Owen}: Game Theory (Academic Press)
\end{description}
\end{english}

\section*{פרטים טכניים}
הקורס מועבר על ידי פרופ' רון הולצמן בטכניון.
שעת הקבלה בימי א' בין השעות 15:30 ו־16:30.

\section*{על תורת המשחקים}

תורת המשחקים פותחה עם כתיבת הספר
\emph{תורת המשחקים והתנהגות כלכלית}
מאת המתמטיקאי ג'ון פון־נוימן והכלכלן אוסקר מורגנשטרן.
עם כן, תורת המשחקים פותחה כתחום מתמטי במחשבה על שימושיה לכלכלה, וכיום הינה כלי מרכזי בחקר כלכלה.
אכן הוענקו פרסי נובל בכלכלה על חקר תורת המספרים, בין השאר עבור המתמטיקאי ג'ון נאש שתרם את עיקרון שיווי המשקל.
במהלך השנים תורת המשחקים חדרה לתחומים נוספים, בינם תורת האבולוציה, ביולוגיה, וחקר מערכות במדעי המחשב.

החלוקה המרכזית בתורת המשחקים היא חלוקה לתחומים של משחקים שיתופיים ושאינם שיתופיים.
בקורס זה נעסוק בשני התחומים. למרות שנראה כי אין קשר בינם, נגלה כי דווקא קיים קשר שכזה.
בצורה כללית ביותר, במשחקים שאינם שיתופיים ההנחה הבסיסית היא שכל אחד מהשחקנים פועל להשגת מטרותיו שלו, והינו אוטונומי לקבלת החלטות לקידום מטרותיו.
במשחקים שיתופיים, השחקנים יוצרים קואליציות בינם ופועלים במשותף כדי לקדם את מטרות אותה קואליציה.

\mainmatter

\part{משחקים שאינם שיתופיים}

\chapter{משחקים דמויי־שחמט}

\section{מאפיינים}

המאפיינים של משחקים דמויי־שחמט הם:%
\newlecture{18 באוקטובר}{2018}
\begin{itemize}
\item יש שני שחקנים
שנקרא להם "לבן" ו־"שחור".
\item שני השחקנים משחקים לסירוגין.
\item בכל שלב שבו שחקן צריך לשחק, הוא יודע את כל מהלך המשחק עד אותו שלב.
\item מהלך המשחק נקבע באופן מלא ע"י החלטות השחקנים: אין צעדי גורל.
\item תוצאת המשחק היא אחת מהבאות.
\begin{itemize}
\item נצחון ללבן.
\item נצחון לשחור.
\item תיקו.
\end{itemize}
\end{itemize}

נתאר משחק דמוי־שחמט בעזרת
stress{עץ משחק}.
השלב ההתחלתי הינו שורש העץ
$r$.
נציין ליד כל קודקוד את זהות השחקן הבוחר את המסע הבא,
$b$
עבור שחור או
$w$
עבור לבן.
הצלעות יתאימו למסעים האפשריים עבורו.
המשחק יכול להסתיים במספר סופי של מסעים, ואם הדבר קורה נקבעת תוצאה (נצחון לאחד הצדדים, או תיקו). לכן, אם הגענו לקודקוד קצה ניתן לרשום עליו את תוצאת המשחק. נסמן
$B$
עבור ניצחון של השחור,
$W$
עבור ניצחון של הלבן, ו־%
$D$
עבור תיקו.

\section{הגדרות}

\begin{definition}
\stress{משחק דמוי־שחמט}
מתואר על ידי עץ משחק ובו המרכיבים הבאים:
\begin{itemize}
\item עץ
$T = \prs{V,E}$ –
כלומר גרף עם קבוצה
$V$
(תיתכן אינסופית) של קודקודים, וקבוצה
$E$
של צלעות, שהוא קשיר וחסר־מעגלים.
\item קודקוד $r$
ב־%
$V$
הנקרא
\stress{שורש}.
אנו חושבים על כל צלע כמכוונת בכיוון המתרחק מן השורש.
\item קבוצת הצלעות היוצאות מקודקוד
$v$
נקראת
\stress{קבוצת המסעים האפשריים ב־%
$v$}.
\item
קודקוד שאין לו עוקבים נקרא
\stress{קודקוד קצה.}
\item
קבוצת קודקודי הקצה תסומן ע"י
$V_{\text{end}}$.
\item
קבוצת הקודקודים שאינם קודקודי־קצה מחולקת לשתי תתי־קבוצות
$V_{\text{white}}, V_{\text{black}}$,
כלומר
\[V \setminus V_{\text{end}} = V_{\text{white}} \cup V_{\text{black}}\]
כאשר
$V_{\text{white}}$
היא קבוצת הקודקודים במרחק זוגי מהשורש ו־%
$V_{\text{black}}$
היא קבוצת הקודקודים במרחק אי־זוגי מהשורש, או להיפף.
קודקודי
$V_{\text{white}}$
נקראים קודקודי ההחלטה של לבן, קודקודי
$V_{\text{black}}$
נקראים קודקודי ההחלטה של שחור.
\item \stress{תחרות} (\textenglish{play})
היא מסלול בעץ המשחק המתחיל בשורש ומקיים אחד משני הבאים.
\begin{enumerate}[label = \alph*.]
\item מסתיים בקודקוד קצה
\item נמשך לבלי סוף
\end{enumerate}
\item קבוצת כל התחרויות תסומן ע"י
$P$.
\item כחלק מתיאור המשחק, נתונה חלוקה
$P = W \amalg B \amalg D$
(זהו סימון לאיחוד זר)
המחלקת את התחרויות לכאלה שהמהוות נצחון ללבן, ניצחון לשחור, או תיקו.
\end{itemize}
\end{definition}
\begin{remark}
חייבת להיות תוצאה למשחק, אך היא יכולה להתקבל לארך אין־סוף מסעים.
לדוגמה, אם שני שחקנים בוחרים ספרות למספר עשרוני בין
$0$
ו־%
$1$.
הלבן מנצח אם המספר הינו רציונלי, והשחור אחרת.
\end{remark}
\begin{definition}
במשחק דמוי שחמט,
\stress{תכסיס} (\textenglish{strategy})
של שחקן הוא כלל האומר לו באיזה מסע לבחור בכל אחד מקודקודי ההחלטה שלו.
כלומר, תכסיס של לבן, זוהי פונקציה
$\sigma \colon V_{\text{white}} \to E$
כך שלכל
$v \in V_{\text{white}}$,
מתקיים כי
$\sigma\prs{v}$
מסע אפשרי ב־%
$v$.
באופן דומה, תכסיס של שחור זוהי פונקציה
$\tau \colon V_{\text{black}} \to E$
כך שלכל
$v \in V_{\text{black}}$,
מתקיים כי
$\tau\prs{v}$
הוא מסע אפשרי ב־%
$v$.
\end{definition}
\begin{example}
כמה תכסיסים יש ללבן בדוגמה שבאיור?
%TODO add fig GT 1
\end{example}
\begin{solution}
ללבן שישה תכסיסים
\begin{align*}
\sigma\prs{r} = \alpha, \quad \sigma\prs{v} = \mu \\
\sigma\prs{r} = \alpha, \quad \sigma\prs{v} = \nu \\
\sigma\prs{r} = \beta, \quad \sigma\prs{v} = \mu \\
\sigma\prs{r} = \beta, \quad \sigma\prs{v} = \nu \\
\sigma\prs{r} = \gamma, \quad \sigma\prs{v} = \mu \\
\sigma\prs{r} = \gamma, \quad \sigma\prs{v} = \nu
\end{align*}
ולשחור עשרים וארבעה תכסיסים
$24 = 2 \cdot 3 \cdot 4$.
נשים לב שזוג
$\prs{\sigma, \tau}$
מורכב מתכסיס
$\sigma$
של לבן ותכסיס
$\tau$
של שחור, קובע לחלוטין את התחרות. בפרט זה קובע לחלוטין את התוצאה.
\end{solution}

\begin{definition}
במשחק דמוי־שחמט, תכסיס של שחקן נקרא 
\stress{תכסיס ניצחון}
אם הוא מבטיח לו ניצחון כנגד כל תכסיס של היריב; הוא נקרא 
\stress{תכסיס תיקו}
אם הוא מבטיח לו לפחות תיקו כנגד כל תכסיס של היריב.
\end{definition}

\begin{theorem}[\textenglish{von-Neumann, 1928}]
יהי $G$ משחק דמוי־שחמט שבו כל התחרויות הן סופיות ובאורך חסום (כלומר, קיים $M$ טבעי כך שהאורך של כל תחרות הוא לכל היותר $M$).
אזי, אחד משלושת הבאים נכון עבור $G$:
\begin{enumerate}
\item ללבן יש תכסיס ניצחון.
\item לשחור יש תכסיס ניצחון.
\item לשני השחקנים יש תכסיסי תיקו.
\end{enumerate} 
\end{theorem}
\begin{example}
נסתכל על העץ באיור%
\newlecture{28 באוקטובר}%
{2018}
%TODO fig GT 2
נסתכל על צומת שכן לעלה בעץ. אם זה צומת לבן, לדוגמה זה שבעומק
$2$,
ויש עלה לבן הסמוך לו, הגעה לצומת זה משמעותה נצחון עבור לבן. לכן אפשר להחליף צומת זה בעלה לבן של העץ.
באותו האופן עבור שחור, למשל בצומת הימני מהשורש. אם כל העלים הסמוכים, הצומת יתאים לתיקו. כך נמשיך באינדוקציה לאחור עד שנקבל רדוקציה שמשחק עם מהלך יחיד שקובע את התוצאה.
\end{example}

\begin{proof}
יהי $G$ משחק כמו במשפט. לכל קדקוד $v$, נגדיר את $G_v$ להיות התת-משחק המתואר ע"י החלק של עץ המשחק $G$ המופיע מהקדקוד $v$ ומעלה. נשים לב שגם $G_v$ הוא משחק דמוי-שחמט המקיים את הנחת המשפט. כמו-כן, נגדיר את עומק הקודקוד
\[
\mrm{depth}(v) = \text{the maximal length of a play in $G_v$}
\]
לצורך ההוכחה נגדיר פונקציה $f \colon V \to \set{W,B,D}$
כך שלכל $v \in V$ יתקיים התנאי הבא 
$(*)_v$:
\begin{itemize}
\item אם $f(v) = W$ אז ללבן יש תכסיס ניצחון ב־$G_v$.
\item אם $f(v) = B$ אז לשחור יש תכסיס ניצחון ב־$G_v$.
\item אם $f(v) = D$ אז לשני השחקנים יש תכסיסי תיקו ב־$G_v$.
\end{itemize}
אם נצליח להגדיר פונקציה כזו, אז התבוננות ב$f(r)$ תראה שבמשחק $G$ עצמו מתקיים א, ב, או ג, כנדרש. אנחנו נגדיר את $f(v)$ ונבדוק את קיום התנאי $(*)_v$ באינדוקציה על 
$\mrm{depth}(v)$.

\begin{description}
\item[בסיס:] $\mrm{depth}(v) = 0$. 
אז $v$ קדקוד־קצה ובתיאור המשחק נתונה התוצאה $W/B/D$ של תחרות המסתיימת בו, ניקח אותה להיות $f(v)$. ברור שמתקיים התנאי $(*)_v$.
\item[צעד:] $\mrm{depth}(v) > 0$. אז $v$ אינו קדקוד־קצה, ולכן יש לו קבוצה לא־ריקה $U$ של עוקבים. 
נשים ♥ שלכל $u \in U$ מתקיים
\[ 
\text{.}\mrm{depth}(u) < \mrm{depth}(v)
\]
לכן נוכל להניח שלכל
$u \in U$, 
$f(u)$ 
כבר הוגדר ומתקיים התנאי. כעת נגדיר את $f(v)$. 
אם
$v \in V_{\mrm{white}}$
נגדיר:
\[f(v) = \fcases{
W & \exists u \in U \colon f(u) = W \\
B & \forall u \in U \colon f(u) = B \\
D & \text{otherwise}
}\]

 
אם
$v \in V_{\mrm{black}}$
נגדיר:
\[f(v) = \fcases{
B & \exists u \in U \colon f(u) = B \\
W & \forall u \in U \colon f(u) = W \\
D & \text{otherwise}
}\]
\end{description}
כעת נראה שמתקיים התנאי $(*)_V$. נניח בלי הגבלת הכלליות ש־ 
$v \in V_{\mrm{white}}$.
נטפל בשלושת המקרים המופיעים בהגדרת
$f(v)$:
\begin{enumerate}
\item $f(v) = W$. 
אז לפי ההגדרה, קיים $u \in U$ קח ש־
$f(u) = W$.
לפי הנחת האינדוקציה, ללבן יש תכסיס נצחון ב־$G_u$, נסמן תכסיס כזה ע"י
$\sigma_u$.
נבנה תכסיס נצחון $\sigma$ ללבן 
ב־
$G_{\sigma}$
באופן הבא:


ב־$v$ לבן יבחר את המסע המוביל ל־$u$. בכל קדקוד החלטה של לבן ב־ $G_u$, 
$\sigma$ 
מתלכד עם 
$\sigma_u$.
בכל שאר קדקודי ההחלטה של לבן של ב־
$G_v$,
נגדיר את $\sigma$ באופן שרירותי. זה אכן מגדיר תכסיס נצחון של לבן ב־
$G_v$.
\item $f(v) = B$. 
לפי ההגדרה, לכל $u \in U$ מתקיים $f(u) = B$. לפי הנחת האינדוקציה, לכל $u$ כזה, יש לשחור תכסיס נצחון ב־$G_u$, נסמנו על ידי 
$\tau_u$.
נבנה תכסיס נצחון $\tau$ לשחור ב־$G_v$ באופן הבא:


בכל קדקוד החלטה של שחור ב$G_u$ כלשהו,
$u \in U$,
$\tau$ 
מתלכד עם
$\tau_u$.
זה אכן מגדיר תכסיס נצחון של שחור ב־$G_v$.
\item $f(v) = D$. 
לפי ההגדרה, לכל $u \in U$ מתקיים 
$f(u) \in \set{B, D}$
וקיים 
$u^{*} \in U$ 
שעבורו
$f(u^*) = D$.
לפי הנחת האינדוקציה, לכל $u \in U$ קיים לשחור תכסיס נצחון או תיקו ב$G_u$, 
נסמן תכסיס כזה ע"י 
$\tau_u$.
כמו־כן, קיים ללבן תכסיס תיקו ב־
$G_{u^*}$,
נסמן תכסיס כזה ע"י
$\sigma_{u^*}$.
ראשית נבנה תכסיס תיקו $\sigma$ 
ללבן ב־$G_{u}$:


ב־$v$ לבן יבחר את המסע המוביל ל־$u^*$. בכל קדקוד החלטה של לבן ב־$G_{u*}$ 
$\sigma$ 
מתלכד עם
$\sigma_{u^*}$.
בכל קדקוד החלטה אחר של לבן ב־$G_v$, $\sigma$ מוגדר באופן שרירותי.
זה אכן מגדיר תכסיס תיקו ללבן ב־$G_v$.

כעת נבנה תכסיס תיקו $\tau$ לשחור ב־$G_v$:

בכל קדקוד החלטה של שחור ב־$G_u$ כלשהו, $u \in U$, 
$\tau$ 
מתלכד עם $\tau_u$.
זה אכן מגדיר תכסיס תיקו לשחור ב־$G_v$.
\end{enumerate}
\end{proof}

\begin{example}[המשחק \textenglish{Chomp}]
עבור שני פרמטרים $m, n$ טבעיים, המשחק
$G_{m, n}$
משוחק על לוח $m \times n$
%TODO add fig GT 2
שבו המשבצת השמאלית התחתונה חסרה.
כל שחקן בתורו (לבן מתחיל) מוחק מן הלוח משבצת שטרם נמחקה, ואת כל המשבצות ברביע שמימינה 
ומעליה. שחקן, שבתורו לשחק לא נותרו משבצות בלוח, מפסיד ויריבו מנצח.
%TODO add fig GT 3
%רון ניצח את אלעד(גל) בצ'ומפ.
%מי רוצה לשחק נגד רון
\begin{claim}
עבור 
$m = n \geq 2$,
ללבן יש תכסיס נצחון ב-
$G_{m, n}$.
\end{claim}
\begin{proof}
לבן יבחר במסע הראשון את המשבצת הסמוכה אלכסונית למשבצת החסרה. כתוצאה מכך, יישארו על הלוח שני טורי משבצות באורכים שווים. מכאן והלאה, על כל מסע של שחור באחד הטורים, לבן יגיב במסע סימטרי לו בטור האחר. לכן, המצב של שני הטורים יהיה סימטרי אחרי כל מסע של לבן, וזה אומר שלבן ינצח.
\end{proof}
\end{example}
\backmatter
\end{document}

