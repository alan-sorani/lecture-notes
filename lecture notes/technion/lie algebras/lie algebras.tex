\documentclass[10pt,a4paper,twoside,openany,hidelinks]{book}
\usepackage{maths}
\usepackage{stylish}

\title{Lecture Notes to a course on Lie Algebras \\ \large{Winter 2018, Technion IIT}}
\author{Lectures by Amos Nevo \\ \large Typed by Elad Tzorani}
\date{\today}

\usepackage{lipsum}
\begin{document}
\frontmatter
\frontpage{lie_symmetry}
\tableofcontents
\countlectures
\newpage

\chapter*{Preface}
\addcontentsline{toc}{chapter}{Preface} \markboth{Preface}{}

\section*{Technicalities}
\addcontentsline{toc}{section}{Technicalities} %\markboth{Technicalities}{}

These aren't formal notes related to the course and henceforward there is \emph{absolutely no guarantee} that the recorded material is in correspondence with the course expectations, or that these notes lack any mistakes.\\
In fact, there probably are mistakes in the notes! I would highly appreciate if any comments or corrections were sent to me via email at \href{mailto:tzorani.elad@gmail.com}{tzorani.elad@gmail.com}.\\
Elad Tzorani.

\section*{Course Literature}
\addcontentsline{toc}{section}{Course Literature} %\markboth{Course Literature}{}

The recommended course literature is as follows.

\begin{description}
\item[Humphreys, James E.:] Introduction to Lie algebras and representation theory.

\item[Jacobson, Nathan:] Lie algebras. New York, 1962.
\end{description}

\mainmatter

\part{Lie Algebras}
\chapter{Preliminaries}

The course will be entirely algebraic, with possibly few examples from analysis.%
\newlecture{October 22}{2018}
This will allow us to discuss issues regarding the algebraic properties of Lie algebras.
We might be interested in infinite-dimensional Lie algebras, but in this course we discuss only finite-dimensional algebras.
In this course one of our main goals is a classification theorem for simple Lie algebras.
We assume knowledge in linear algebras and specifically bilinear forms.

\section{Basic definitions}

Let $\FF$ be a field, and $V$ a finite-dimensional vector-space over $\FF$.

\begin{definition}
$V$ is a \stress{generalised $\FF$-algebra} if it comes with a map $m \colon V \times V \to V$ which is bilinear.
\begin{align*}
m\prs{v_1 + v_2, w} &= m\prs{v_1, w} + m\prs{v_2, w} \\
m\prs{v, w_1 + w_2} &= m\prs{v, w_1} + m\prs{v, w_2} \\
m\prs{av, bw} &= abm\prs{v,w}
\end{align*} 
\end{definition}
\begin{example}
Let $V$ be an associative algebra.
Here $m$ is an associative operation which is left and right distributive on addition in $v$.
Equivalently: If we denote $m\prs{v,w} = v \odot w$ then
\begin{align*}
\prs{v \odot w} \odot u &= v \odot \prs{w \odot u} \\
v \odot \prs{u+w} &= v\odot u + v\odot w \\
\prs{u+w}\odot u &= u\odot v + w\odot v
\end{align*}
\end{example}
\begin{remark}
Here associativity means the following.
\[m\prs{v, m\prs{w,u}} = m\prs{m\prs{v,w},u}\]
\end{remark}
\begin{examples}
\begin{enumerate}
\item Every field $k$ is an $\FF$–algebra over any subfield $\FF$.

\item $M_n\prs{k}$ is an $\FF$–algebra.

\item $P_n$, polynomials over $k$ of degree smaller or equal to $n$, is an $\FF$-algebra.
\end{enumerate}
\end{examples}

\begin{definition}
A Lie algebra $L$ over $\FF$ is an $\FF$-algebra, so $\exists m \colon L \times L \to L$, which generally need not be associative, but instead satisfies the following \stress{Jacobi identity},
\begin{align*}
m\prs{X, m\prs{Y,Z}} + m\prs{Z, m\prs{X,Y}} + m\prs{Y,m\prs{Z,X}} = 0
\end{align*}
and additionally, antisymmetry of the multiplication
\[m\prs{X,Y} = -m\prs{Y,X} \text{.}\]
If $\mathrm{char}\FF = 2$ we require $m\prs{X,X} = 0$.
\end{definition}
\begin{notation}
The "multiplication" in $L$ is called \stress{bracket}, and denoted $m\prs{X,Y} = \brs{X,Y}$ ($X$ bracket $Y$).
\end{notation}
\begin{remark}
In these terms we write the Jacobi identity as follows.
\[\brs{X,\brs{Y,Z}} + \brs{Z,\brs{X,Y}} + \brs{Y,\brs{Z,X}} = 0\] 
\end{remark}
\begin{definition}
A \stress{Lie algebra} $L$ is a vector space over $\FF$ with a bilinear map $\brs{,} \colon L \times L \to L$, which is anti-symmetric and satisfies the Jacobi identity.
\end{definition}
\begin{definition}
Given a Lie algebra $L$, a vector subspace $L_0 \subseteq L$ is called a sub-Lie-algebra if it is closed under brackets. I.e.
\[X,Y \in L_0 \implies \brs{X,Y} \in L_0 \text{.}\]
\end{definition}
\begin{examples}
\begin{enumerate}
\item \emph{Abelian Lie algebras:} The bracket is the zero form.
\[\forall X,Y \in L\colon \brs{X,Y} = 0\]
\end{enumerate}
\end{examples}
\begin{example}
$\FF$ is itself a Lie algebra as well as any $\FF$-vector space $V$ under the bracket \[\forall u,v \in V\colon \brs{u,v} = 0\text{.}\]
\end{example}
\begin{example}
Let $A$ be any associative $\FF$-algebra, and define on $A$ \emph{another} bilinear operation, namely
\[\brs{a,b} = ab - ba \text{.}\]
This is called \stress{the commutator of $a$ and $b$}.
Then $\brs{,} \colon A \times A \to A$.
\begin{exercise}
This bracket satisfies the Jacobi identity, and is anti-symmetric.
\end{exercise}
Given a solution to this exercise, $\prs{A, \brs{,}}$ is a Lie algebra.
\\
In particular, $M_n\prs{k}$ is a Lie algebra under the bracket $\brs{A,B} = AB - BA$.
This algebra is \emph{very important} and is denoted $\Gg \Ll_n \prs{k}$.
\end{example}
\begin{exercise}
Consider the subspace \[\set{A \in \Gg \Ll_n\prs{k}}{\mathrm{tr} A = 0} \subseteq \Gg \Ll_n\prs{k}\text{.}\] Is the subspace a Lie algebra? Yes! Since for any $A,B \in \Gg \Ll_n\prs{k}$ we have that $\mathrm{tr}\prs{AB} = \mathrm{tr}\prs{BA}$, we get that $\mathrm{tr}\brs{A,B} = 0$.
The sub-Lie-algebra of zero-trace matrices is denoted $\Ss\Ll_n\prs{k}$.
\end{exercise}
\begin{exercise}[Lie algebras associated with bilinear forms]
Let $V$ be a vector space over $\FF$, and $B \colon V \times V \to \FF$ be a bilinear form.
Assume $B$ is anti-symmetric. Define \[L_B = \set{X \in \mathrm{End}\prs{V}}{B\prs{Xv,w} = -B\prs{v,Xw}}\text{.}\]
Check that $L_B$ is a vector subspace of $\mathrm{End}\prs{V}$.
Consider the bracket operation on $\mathrm{End}\prs{V}$, defined by $\brs{T,S} = TS - ST$.
Is $L_B$ closed under brackets?
\end{exercise}
\begin{solution}
We compute as follows.
\begin{align*}
B\prs{\brs{X,Y}v, w} &= B\prs{\prs{XY - YX}v, w} \\
&= B\prs{XY v, w} - B\prs{YXv,w} \\
&= -B\prs{Yv,Xw} + B\prs{Xv,Yw} \\
&= B\prs{v,YXw} -B\prs{v,XYw} \\
&= B\prs{v, \prs{YX - XY}w} \\
&= -B\prs{v, \brs{X,Y}w}
\end{align*}
In conclusion, $L_B$ is a sub-Lie-algebra of $\mathrm{End}\prs{V}$, the Lie algebra associated with the form $B$.
\end{solution}
\begin{exercise}
Let $S$ be a symmetric bilinear form, and let \[L_S = \set{X \in \mathrm{End}\prs{V}}{S\prs{Xv, w} = -S\prs{v,Xw}}\text{.}\]
Then again, $L_S$ is a Lie sub-algebra.
\end{exercise}
\begin{examples}[Sub-algebras of $\Gg\Ll_n\prs{\FF}$]
\begin{enumerate}
\item
\[\tT\prs{n,\FF} = \set{\pmat{a_{11} & & a_{i,j} \\ & \ddots & \\ 0 & & a_{nn}}}{a_{ij} \in \FF}\]
is closed under the bracket operation, for if $A,B \in \tT\prs{n,\FF}$ then $AB \in \tT\prs{n,\FF}$ and so $AB - BA \in \tT\prs{n,\FF}$.
\item \[\nN\prs{n,\FF} = \set{\pmat{0 & & a_{i,j} \\ & \ddots & \\ 0 & & 0}}{a_{ij} \in \FF}\]
is a Lie sub-algebra of $\tT\prs{n,\FF}$.
\item \[\dD\prs{n,\FF} = \set{\pmat{a_1 & & 0 \\ & \ddots & \\ 0 & & a_n}}{a_i \in \FF}\]
an abelian sub-algebra. 
\end{enumerate}
\end{examples}
\section{Structure constants}
Let $L$ be a Lie algebra and let $X_1, \ldots, X_n$ be a basis of $L$, Then the bracket operation is completely determined by the structure constants with respect to the basis.
\[\brs{X_i, X_j} = \sum_{k=1}^n c_k^{i,j} X_k\]
The \stress{structure constants} $c_k^{i,j}$ contain full information on the bracket operation of course. These satisfy two properties associated with anti-symmetry and the Jacobi identity of the brackets.
The property associated to anti-symmetry is $c_k^{i,j} = -c_k^{j,i}$. The other property (associated to the Jacobi identity) is left as an \textbf{\textrm{Exercise}}.
\begin{example}
\[\Gg\Ll_n\prs{\FF} = \mathrm{span}\set{E_{i,j}}{1 \leq i,j\leq n}\]
In the basis $E_{ij}$ the structure constants are very simple. We have the following.
\[\brs{E_{i,j}, E_{k,l}} = \delta_{j,k}E_{i,l} - \delta_{l,i}E_{k,j}\]
Hence all the structures constants are $1$ or $-1$.
\end{example}
\begin{definition}
Let $L_1, L_2$ be Lie algebras. A \stress{Lie algebra homomorphism} between $L_1$ and $L_2$ is a linear map $T \colon L_1 \to L_2$ satisfying
\[T\brs{X,Y} = \brs{TX, TY}\text{.}\]
\end{definition}
\begin{definition}
Let $L$ be a Lie algebra. A sub-space $I \subseteq L$ is called a \stress{Lie-ideal} of $L$ if for all $X \in L$ and $Y \in I$, we have that $\brs{X,Y} \in I$. This is written also by 
\[\brs{L,I} = \mathrm{span}\set{\brs{X,Y}}{X \in L, Y \in I} \subseteq I \text{.}\]
\end{definition}
\begin{definition}
Let $L$ be a Lie algebra and $L_0 \subseteq L$ be a sub-space. The \stress{Lie normaliser} of $L_0$ is
\[N\prs{L_0} = \set{X \in L}{\brs{X,L_0} \subseteq L_0}\text{.}\]
The \stress{Lie centraliser} of $L_0$ is
\[Z\prs{L_0} = \set{X \in L}{\brs{X,L_0} = 0} \text{.}\]
\end{definition}
\begin{definition}
Let $L$ be a Lie algebra. If $\brs{X,Y} = 0$ one says that $X$ and $Y$ commute.
We sometimes refer to the bracket as the commutator.
\end{definition}
\begin{example}
Two sub-spaces $L_1, L_2 \subseteq L$ of a Lie algebra commute if their commutators are zero. I.e.
\[\brs{L_1, L_2} = 0 \text{.}\]
\end{example}
\begin{remark}
Although we have linearity of the bracket, we do need to take the span in the above example. If we take $X,X' \in L_1$ and $Y,Y' \in L_2$ we can't always express $\brs{X,Y} + \brs{X',Y'}$ as a bracket of two elements, although it certainly is in the span.
\end{remark}
\section{Linear representations}
\begin{definition}
A \stress{linear representation} of a Lie algebra $L$ over $\FF$ is a Lie-algebra homomorphism $T \colon L \to \mathrm{End}(V) \cong \Gg\Ll_n\prs{\FF}$ where $V$ is an $n$-dimensional vector space over $\FF$.
\end{definition}
\begin{remark}
The bracket operation on $\mathrm{End}(V)$ is the usual one, namely $\brs{A,B} = AB - BA$.
\end{remark}
Let us define another large collection of Lie algebras. First, let $A$ be a generalised $\FF$–algebra, and denote $m\prs{a,b} = a \odot b$.\\
\begin{definition}
A \stress{derivation} of the generalised algebra $A$ is a linear map $\delta \colon A \to A$ satisfying the following property.
\begin{align*}
\delta \prs{ a \odot b} = \delta\prs{\alpha} \odot b + a \odot \delta\prs{b}
\end{align*}
\end{definition}
\begin{definition}
\[\mathrm{Der}\prs{A} \ceq \set{\delta \in \mathrm{End}\prs{A}}{\text{$\delta$ is a derivation.}}\]
\end{definition}
\begin{remark}
$\mathrm{Der}(A)$ is clearly a linear sub-space of $\mathrm{End}(A)$.
Now, if $\delta_1$ and $\delta_2$ are derivations, $\delta_1 \circ \delta_2$ is \emph{not} a derivation, usually.
But, $\brs{\delta_1, \delta_2} = \delta_1 \circ \delta_2 - \delta_2 \circ \delta_1$ \emph{is} in fact a derivation.
\end{remark}
\begin{conclusion}
$\mathrm{Der}(A)$, with the bracket inherited from $\mathrm{End}(A)$ is a Lie algebra.
\end{conclusion}
\begin{proof}
We compute the following.
\begin{align*}
\brs{\delta_1, \delta_2}\prs{a \odot b} &= \prs{\delta_1 \circ \delta_2 - \delta_2 \circ \delta_1}\prs{a \odot b} \\
&= \delta_1 \circ \delta_2\prs{a \odot b} - \delta_2 \circ \delta_1 \prs{a \odot b} \\
&= \delta_1 \prs{\delta_2\prs{a}\odot b + a\odot \delta_2\prs{b}} - \delta_2 \prs{\delta_1\prs{a}\odot b + a\odot\delta_1\prs{b}} \\
&= \delta_1 \delta_2 \prs{a} \odot b + \delta_2\prs{a} \odot \delta_1\prs{b} + \delta_1\prs{a}\odot \delta_2\prs{b} + a\odot \delta_1\delta_2\prs{b} \\ &- \prs{\delta_2 \delta_1\prs{a} \odot b + \delta_1 \prs{a} \odot \delta_2 \prs{b} + \delta_2\prs{a} \odot \delta_1\prs{b} + a\odot \delta_2 \delta_1\prs{b}} \\
&= \prs{\delta_1 \delta_2 - \delta_2 \delta_1} \prs{a} \odot b + a\odot \prs{\delta_1 \delta_2 - \delta_2 \delta_1}\prs{b}
\end{align*}
\end{proof}
\begin{example}
\begin{enumerate}
\item If $A$ is an associative algebra, then $\mathrm{Der}(A)$ is a Lie algebra, $\mathrm{Der}(A) \subseteq \mathrm{End}(A)$. $\mathrm{Der}(A)$ is a sub-Lie-algebra of $\mathrm{End}(A)$ under bracket of linear transformations.
\item A Lie algebra is a generalised algebra and so $\mathrm{Der}(L)$ is another Lie algebra.
\end{enumerate}
\end{example}
\begin{fact}[important]
There is a very natural collection of derivations of any Lie algebras.
For each $x \in L$, let us define a linear transformation denoted $\mathrm{ad}(x) \colon L \to L$ (this stands for "adjoint") via
$\mathrm{ad}(x)(y) = \brs{x,y}$. (This is linear from the bi-linearity of the bracket)
In fact, $\mathrm{ad}(x)$ is a derivation of $L$. Namely,
\[\mathrm{ad}(x)\prs{\brs{y,z}} = \brs{\mathrm{ad}(x)y,z} + \brs{y, \mathrm{ad}(x)(z)}\text{.}\]
Indeed,
\begin{align*}
\mathrm{ad}(x)\prs{\brs{y,z}} &= \brs{x,\brs{y,z}} \\
&= \brs{\mathrm{ad}(x)y, z} + \brs{y, \mathrm{ad}(x)z} \\
&= \brs{\brs{x,y},z} + \brs{y,\brs{x,z}}
\end{align*}
which is an identity as a consequence of the Jacobi identity.
\end{fact}
\begin{conclusion}
The set $\set{\mathrm{ad}(x)}{x \in L} \subseteq \mathrm{Der}(L)$ is a sub-algebra.
We have the map $x \mapsto \mathrm{ad}(x)$ which is obviously linear (from bi-linearity of the bracket).
So, $\mathrm{ad}(L) \ceq \set{\mathrm{ad}(x)}{x \in L}$ is a linear sub-space.
In fact it is a Lie sub-algebra of $\mathrm{Der}(L)$.
\end{conclusion}
\begin{proof}
We have to show that $\brs{\mathrm{ad}(x),\mathrm{ad}(y)} = \mathrm{ad}(x)\mathrm{ad}(y) - \mathrm{ad}(y)\mathrm{ad}(x)$ is in the space $\mathrm{ad}(L)$. But, actually $\brs{\mathrm{ad}(x), \mathrm{ad}(y)} = \mathrm{ad}\brs{x,y}$, as the following proposition states.
\begin{proposition}
$\mathrm{ad} \colon L \to \mathrm{Der}(L)$
is a Lie algebra homomorphism.
\end{proposition}
\begin{proof}
Let us compute.
\begin{align*}
\brs{\mathrm{ad}(x), \mathrm{ad}(y)}\prs{z} &= \mathrm{ad}\prs{x} \mathrm{ad}\prs{y}\prs{z} - \mathrm{ad}\prs{y}\mathrm{ad}\prs{x}\prs{z} \\
&= \brs{x,\brs{y,z}} - \brs{y,\brs{x,z}} \\
&\stackrel{\star}{=} \brs{\brs{x,y},z} \\
&= \mathrm{ad}\brs{x,y}\prs{z}
\end{align*}
where the $\star$ is given from the Jacobi identity.
\end{proof}
In conclusion, $\mathrm{Der}(L)$ is a Lie sub-algebra of $\mathrm{End}(L)$ under bracket, and $\mathrm{ad} \colon L \to \mathrm{Der}(L) \subseteq \mathrm{End}(L)$ is a linear representation of the Lie algebra $L$ with the image being $\mathrm{ad}(L) = \set{\mathrm{ad}(x)}{x \in L}$.
\end{proof}
\begin{example}
Given $L_0 \subseteq L$ a sub-space. Then
$N\prs{L_0} = \set{x}{\brs{x,L_0} \subseteq L_0}$ is the set of elements $x$ such that the linear transformation $\mathrm{ad}\prs{x}$ leaves the subspace $L_0$ invariant. $N_L\prs{L_0}$ is a Lie sub-algebra, and if $L_0$ is a Lie sub-algebra, then $L_0$ is an ideal of $N_L\prs{L_0}$.
\end{example}
\begin{example}
The condition $\brs{X,Y} = 0$ means $Y \in \ker\prs{\mathrm{ad}(x)}$ or equivalently $x \in \ker\prs{\mathrm{ad}(y)}$. Therefore
\begin{align*}
Z\prs{L_0} &= \set{x \in L}{\brs{x,L_0} = 0} \\
&= \set{x \in L}{L_0 \subseteq \ker\prs{\mathrm{ad}(x)}} \text{.}
\end{align*}
$Z\prs{L_0}$ is a Lie sub-algebra of $L$, the Lie sub-algebra of elements commuting with every $x \in L_0$.
\end{example}
\begin{remark}
If $L_0 \subseteq L$ is a Lie sub-algebra, then $N\prs{L_0}$ is the largest sub-algebra such that $L_0$ is is an ideal in it.
\end{remark}

\begin{remark}
$Z_L(L)$ is the center of $L$, and an ideal.%
\newlecture{October 29}{2018}
Indeed, if $z \in Z\prs{L}$, and $x \in L$, then $\ad\brs{x,z} = \ad x \ad z - \ad z - \ad x$ and $L \subseteq \ker \ad z$, so $L \subseteq \ker \ad \brs{x,z}$, so $\brs{x,z} \in Z\prs{L}$ and $Z(L)$ is an ideal.
\end{remark}
\section{Sub-algebras and ideals}
\begin{remark}
\begin{enumerate}
\item If $L_1$ and $L_2$ are Lie sub-algebras, then $L_1 + L_2$ generally \emph{is not}!
\item Suppose $I = L_1$ is an ideal and $L_2$ a sub-algebra. Then $I + L_2$ is a sub-algebra.
\item If $L_1 = I$ and $L_2 = J$ are ideals, then the Lie sub-algebra $I+J$ is an ideal.
Indeed $\brs{x,i} \in I$ and $\brs{x,j} \in J$ for all $j$, so $\brs{x,I+J} \subseteq I+J$.
\end{enumerate}
\end{remark}
\begin{definition}
The \stress{commutator} of two sub-algebras $L_1,L_2$ is defined to be
\[\mrm{Span}\set{\brs{X,Y}}{X \in L_1, Y \in L_2} \text{.}\]
\end{definition}
\begin{remark}
The commutator of two sub-algebras \emph{is not} in general a sub-algebra. Generally $\brs{\brs{X,Y},\brs{X',Y'}}$ isn't in $\brs{L_1, L_2}$ if $X,X' \in L_1$ and $Y,Y' \in L_2$.
Let \[\sum_{i=1}^n \brs{X_i, Y_i} \in \brs{L_1, L_2}\] and \[\sum_{j=1}^m \brs{X_j',Y_j'} \in \brs{L_1, L_2}\text{.}\]
Then
\begin{equation}\label{commut_subalgebra}
\brs{\sum_{i=1}^n \brs{X_i,Y_i} \sum_{j=1}^m \brs{X_j',Y_j'}} = \sum_{\substack{i=1 \\ j=1}}^{\substack{n \\ m}} \brs{\brs{X_i,Y_i},\brs{X_j',Y_j'}}\text{.}
\end{equation}
\begin{enumerate}
\item If $L_1 = I$ is an ideal, then $\brs{I,L_2} \subseteq I$, is a sub-space of $I$.
\item If $L_1 = I$ and $L_2 = J$ are ideals, then $\brs{I,J} \subseteq I \cap J$, and it is an ideal of $L$.
Equation \ref{commut_subalgebra} shows that $\brs{I,J}$ is indeed a sub-algebra. Now, let $\brs{i,j} \in \brs{I,J}$, and let $x \in L$. We should show that $\brs{x,\brs{i,j}} \in \brs{I,J}$ which is sufficient for the span.
Now
\begin{align*}
\brs{x,\brs{i,j}} \stackrel{\text{Jacobin identity}}{=} \brs{\brs{x,i},j} + \brs{i,\brs{x,j}} = \brs{i',j} + \brs{i,j'} \in \brs{I,J}
\end{align*}
as required.
\end{enumerate}
\end{remark}
\begin{conclusion}
$I+J$ and $\brs{I,J}$ are ideals if $I$ and $J$ are.
\end{conclusion}
\begin{remark}
In general $\brs{I,J} \subseteq I \cap J$, but the inclusion may be strict.
\begin{examples}
\begin{enumerate}
\item Take $L$ an abelian Lie algebra and $I,J$ any two sub-spaces which are both sub-algebras, and ideals. Then $\brs{I,J} = 0$, but $I \cap J$ may be large.
\item Take $L$ a Lie algebra of upper-triangular matrices, and $I=J$ the ideal of strict upper-triangular matrices. Then $\brs{I,I}$ contains matrices that have zero entries in the diagonal above the main diagonal, hence $\brs{I,I} \subsetneq I\cap J = I$.
\end{enumerate}
\end{examples}
\end{remark}
\begin{definition}
If $\brs{I,J} = 0$, we say that $I$ and $J$ \stress{commute}.
\end{definition}
\begin{remark}
$L$ is an ideal of itself, so $\brs{L,L} = \mrm{Span}\set{\brs{X,Y}}{X,Y \in L}$ is also an ideal, \stress{the commutator ideal of $L$}.
\end{remark}
\begin{definition}
$L$ is \stress{abelian} if $\brs{L,L} = 0$.
\end{definition}
\begin{definition}
$L$ is \stress{perfect} if $\brs{L,L} = L$.
\end{definition}
\begin{definition}
$L$ is called a \stress{simple Lie-algebra} if $\dim L > 1$ and $L$ has no non-trivial ideals.
\end{definition}
\begin{exercise}
A simple Lie algebra is in particular perfect.
\end{exercise}
\begin{proposition}
If $\phi \colon L \to L'$ is a Lie-algebra homomorphism, then $\ker \phi$ is an ideal.
\end{proposition}

\begin{definition}
For any ideal $I \triangleleft L$, the factor vector space $\quot{L}{I} = \set{\ell + I}{\ell \in L}$ has a structure of a Lie algebra, given by the following.
\begin{align*}
\brs{x+I, y+I}_{\quot{L}{I}} \ceq \brs{x,y}_L + I
\end{align*}
\end{definition}
\begin{remark}
The above is well defined since
\[\brs{x+i, y+i'} = \brs{x,y} + \brs{i,y} + \brs{x,i'} + \brs{i,i'} \equiv \brs{x,y} \mod{I}\text{.}\]
The identities for Lie algebras follow immediately from those on $L$. 
\end{remark}
\begin{theorem}[1\textsuperscript{st} homomorphism theorem]
\begin{align*}
\pi \colon L &\to \quot{L}{I} \\
x &\mapsto x+I
\end{align*}
is a surjective Lie-algebra homomorphism, and $\ker \pi = I$.
\end{theorem}
\begin{theorem}[2\textsuperscript{nd} homomorphism theorem]
If $I$ and $J$ are ideals of $L$, and $I \subset J$, then the map
\begin{align*}
\phi \colon \quot{L}{I} &\to \quot{L}{J} \\
x+I &\mapsto x+J 
\end{align*}
is a well-defined Lie-algebra epimorphism.
We have from the first homomorphism theorem that
\[\quot{\quot{L}{I}}{\ker \phi} \cong \quot{L}{J}\]
and
\[\ker \phi = \quot{J}{I}\]
therefore
\[\quot{\quot{L}{I}}{\quot{J}{I}} \cong \quot{L}{J}\text{.}\]
\end{theorem}
\begin{theorem}[3\textsuperscript{rd} homomorphism theorem]
Given any two ideals $I,J$, their intersection $I \cap J$ is an ideal of $L$ and we have a map
\begin{align*}
\psi \colon I &\to \quot{I+J}{J} \\
i &\mapsto i+J \phantom{\quot{}{J}}\text{.}
\end{align*}
This is a Lie-algebra homomorphism which is obviously surjective, with kernel $I \cap J$, hence
\[\quot{I}{I \cap J} \cong \quot{I+J}{J}\]
with Lie-algebra homomorphism induced by $\psi$.
\end{theorem}
\begin{remark}
If $L_0$ is an arbitrary Lie sub-algebra of $L$, and $J \triangleleft L$, then
$J \cap L_0 \triangleleft L_0$ and $J \triangleleft L_0 + J$, and the Lie algebras $\quot{L_0 + J}{J}$ and $\quot{L_0}{L_0 \cap J}$ are isomorphic under the canonical map $\psi$.
\end{remark}

\chapter{Structure of Lie algebras}
\section{Nilpotent Lie algebras}
\begin{definition}
The commutator ideal $\brs{L,L}$ is denoted $L^{\prs{1}}$.
Similarly we denote $L^{\prs{n}} = \brs{L^{\prs{n-1}},L}$, which is an ideal of $L$.
\end{definition}
\begin{remark}
The above gives a descending chain
\[L = L^{\prs{0}} \supseteq \supseteq L^{\prs{1}} \supseteq L^{\prs{2}} \supseteq \ldots\]
and since $\dim L < \infty$, this sequence has to stabilise.
It is however possible that $\brs{L,L}=0$, if $L$ is abelian, or that $\brs{L,L} = L$, if $L$ is perfect.
\end{remark}
\begin{definition}
If $L^{\prs{n}} = 0$ for some $n$, $L$ is called a \stress{nilpotent Lie algebra}
.If $L^{\prs{n}} = 0$ and $L^{\prs{n-1}} \neq 0$, we call $n-1$ the \stress{index of nilpotency}.
\end{definition}
\begin{note}
In some books $n$ itself is called the index of nilpotency.
\end{note}
\begin{definition}
The sequence of ideals $L^{\prs{n}}$ is called \stress{the descending central series} of $L$.
\end{definition}
\begin{remark}
$L^{\prs{k}} \triangleleft L$ and hence $L^{\prs{l}} \triangleleft L^{\prs{k-1}}$. Also $\quot{L^{\prs{k-1}}}{L^{\prs{k}}}$ is an \emph{abelian} algebra
since $L^{\prs{k}} = \brs{L^{\prs{k-1}}, L} \supseteq \brs{L^{\prs{k-1}}, L^{\prs{k-1}}}$ and in general an ideal $I \triangleleft M$ is such that $\quot{M}{I}$ is abelian if and only if $I \supseteq \brs{M,M}$.
\end{remark}
\backmatter
\end{document}