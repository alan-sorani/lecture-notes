\documentclass[a4paper,10pt,twoside,openany]{book}

\usepackage[lang=hebrew]{maths}
\usepackage{hebrewdoc}
\usepackage{stylish}
\usepackage{lipsum}

\title{סיכומי הרצאות ותרגולים במבוא לתורת המספרים \\ \large{חורף 2018, הטכניון}}
\author{הרצאות ותרגולים של פרופסור משה ברוך \\ \large סוכמו על ידי אלעד צורני}
\date{\today}

\begin{document}
\frontmatter
\frontpage{lehmer_sieve}{0.8\textwidth}{נפת להמר.}
\tableofcontents
\countlectures
\counttutorials
\newpage

\chapter*{הקדמה}

\section*{הבהרה}

סיכומי הרצאות אלו אינם רשמיים ולכן אין
\emph{כל הבטחה}
כי החומר המוקלד הינו בהתאמה כלשהי עם דרישות הקורס, או שהינו חסר טעויות.
\\
להיפך, ודאי ישנן טעויות בסיכום! אעריך אם הערות ותיקונים ישלחו אלי בכתובת דוא"ל
\textenglish{\href{mailto:tzorani.elad@gmail.com}{tzorani.elad@gmail.com}}.\\
אלעד צורני.

\section*{ספרות מומלצת.}

הספרות המומלצת עבור הקורס הינה כדלהלן.

\begin{english}
\begin{description}
\item[Ireland and Rosen:] A classical introduction to modern number theory
\end{description}
\end{english}

\section*{סילבוס}
חוגים אוקלידיים, משפט השארית הסיני ושלמים גאוסים. שרשים פרימיטיביים, הדדיות ריבועית, סכומי גאוס, סכומי יעקובי. הדדיות מסדר שלוש, הדדיות מסדר ארבע, מספרים אלגבריים ושדות ריבועיים.\\
הסילבוס יכלול את הפרקים הבאים מספר הקורס: 1,34,5,6,8,9.

\section*{דרישות קדם}
דרישת הקורס העיקרית הינה ידע של קורס מבוא בחבורות. נשתמש גם בידע מקורס בסיס בחוגים על חוגים אוקלידיים, ונניח את ההגדרות הבסיסיות. נחזור על נושא זה בתחילת הקורס.

\section*{ציון:}

\begin{enumerate}
\item בוחן אמצע: 20\% מגן.
\item שאלת תרגילי בית בבוחן
5\%
מגן.
\item שאלת תרגילי בית במבחן
5\%
מגן.
\item מבחן סופי.
\end{enumerate}

\mainmatter

\chapter{מבוא}
\section{רקע היסטורי}

בין שנת 1640 לשנת 1654, מתמטיקאי בשם פרמה%
\footnote{Pierre de Fermat}
הסתכל על מספר שאלות בנוגע למספרים.%
\newlecture{24 באוקטובר}{2018}
\begin{question}
אילו ראשונים
$p$
הם מהצורה
\begin{enumerate}
\item $x^2 + y^2$
\item $x^2 + 2y^2$
\item $x^2 + 3y^2$
\end{enumerate}
כאשר
$x,y \in \ZZ$?
\end{question}
\begin{solution}
\begin{enumerate}
\item פרמה ניסח את המשפט הבא
\begin{theorem}
יהא
$p$
ראשוני אי־זוגי. קיימים שלמים
$x,y$
ש־%
$p= x^2 + y^2$
אם ורק אם
$p \equiv 1 \mod{4}$.
\end{theorem}
\item נסו למצוא חוקיות לבד.

\item
\begin{theorem}[פרמה]
 יהא
$p \neq 3$
ראשוני. קיימים
$x,y \in \ZZ$
כך ש־%
$x^2 + 3y^2 = p$
אם ורק אם
$p \equiv 1 \mod{3}$.
\end{theorem}
\end{enumerate}
\end{solution}
בין השנים 1729 ו־1772 אוילר%
\footnote{Leonhard Euler}
את שלושת המשפטים של פרמה.
אוילר הוכיח את המשפטים בשני שלבים, הורדה
\textenglish{descent}
והדדיות
\textenglish{Reciprocity}.
אנחנו נשתמש בחוגים אוקלידיים עבור השלב הראשון, על מנת לפשט את ההוכחה.

\section{חוגים וחוגים אוקלידיים}

\subsection{חוגים כלליים}

ניתן מספר דוגמאות לחוגים.

\begin{examples}
\begin{itemize}
\item $\ZZ$
\item $M_n\prs{R}$
חוג מטריצות
$n \times n$
מעל חוג
$R$.
\item חוג פולינומים
$R\brs{X}$
מעל חוג
$R$.
\end{itemize}
\end{examples}
בקורס זה נניח כי כל החוגים הינם קומוטטיבים עם יחידה וללא מחלקי אפס (כלומר אם $ab = 0$ אז $a=0$ או $b=0$).

\begin{definition}
חוג עם התכונות הנ"ל נקרא
\stress{תחום שלמות}.
\end{definition}
יהא
$R$
חוג ויהיו
$a,b \in R$.

\begin{definition}
נאמר כי
$a$
\stress{מחלק את}
$b$
אם קיים
$d \in R$
עבורו
$ad = b$.
אם כן, נסמן
$a \mid b$.
\end{definition}
\begin{definition}
$a$
הפיך אם
$a \mid 1$.
\end{definition}
\begin{definition}
$a\neq 0$
שאינו הפיך הוא
\stress{ראשוני}
ב־%
$R$
אם
$a \mid bc$
גורר
$a \mid b$
או
$a \mid c$.
\end{definition}
\begin{definition}
$a\neq 0$
שאינו הפיך נקרא
\stress{אי־פריק}
אם
$a=bc$
גורר כי
$b$
הפיך או
$c$
הפיך.
\end{definition}

\begin{definition}
$a \equiv b \mod{c}$
אם
$c \mid \prs{b-a}$.
\end{definition}
\begin{claim}
אם
$a$
ראושני, הוא אי פריק.
\end{claim}
\begin{proof}
יהי
$a$
ראשוני ונכתוב
$a=bc$.
אז
$a \mid bc$.
לכן
$a \mid b$
או
$a \mid c$.
אם
$a \mid b$
קיים
$d$
עבורו
$b = ad$.
אז
$a = adc$.
לכן
$a\prs{1-dc} = 0$
ולכן
$dc = 1$.
לכן $c$ הפיך.
אחרת,
$a \mid c$
ונקבל באותו אופן כי
$b$
הפיך.
\end{proof}

\subsection{חוגים אוקלידיים}

\begin{definition}
חוג
$R$
יקרא
\stress{חוג אוקלידי}
אם קיימת פונקצייה
$N \colon R \setminus\set{0} \to \NN_{0}$
כך שמתקיימות שתי התכונות הבאות.

\begin{enumerate}
\item
אם
$a,b \in R$
שונים מאפס, קיימים
$q,r \in R$
כך שמתקיים
$r = 0$
או
$N(r) < N(a)$
וגם
$b = qa + r$.
\item
אם
$a \neq 0$
וגם
$a=bc$
כאשר
$b,c$
אינם הפיכים, אז
$N(c),N(b)<N(a)$.
\end{enumerate}
\end{definition}

\begin{remark}
התכונה השנייה בהגדרה איננה הכרחית.
\end{remark}

\begin{examples}
\begin{enumerate}
\item $\ZZ$
עם
$N(x) = \abs{x}$.
\item $k\brs{X}$
פולינומים מעל שדה, עם
$N\prs{p(x)} = \deg\prs{p}$.
\end{enumerate}
\end{examples}
\begin{remark}
חלוקה בחוג אוקלידי איננה יחידה.
אם נדרוש גם
$N(0) \leq N(r) < N\prs{a}$
נקבל כי החלוקה תהיה יחידה.
\end{remark}

נניח בקורס כי
$\abs{r} \leq \frac{\abs{a}}{2}$.
אפשר לדרוש זאת במקרה
$r \geq 0$
כי אם
$b = qa + r$
נחליף את
$r$
ב־%
$r-a$.
נקבל
$b = \prs{q+1}a + \prs{r-a}$
ואז
\begin{align*}
\text{.} \abs{r-a} = \abs{a-r} = \abs{a} - \abs{r} \leq \abs{a} - \abs{\frac{a}{2}} = \abs{\frac{a}{2}}
\end{align*}
באופן דומה נוכיח עבור המקרה
$r < 0$.

\begin{proposition}
בחוג אוקלידי
$R$,
כל אידאל הינו ראשי. כלומר,
אם
$I \leq R$
אידאל, הוא מהצורה
$I = \prs{d} = dR = \set{dr}{r \in R}$
עבור
$d \in R$.
\end{proposition}
\begin{proof}
נמצא ב־%
$I$
איבר
$d$
עם נורמה מינימלית (כתרגיל) ואז נראה
$I = \prs{d}$.
ניקח איבר
$a \in I$,
נכתוב
$a = qd + r$.
אז
$r = 0$
כי לא ייתכן
$N(r) < N(d)$.
\end{proof}
\begin{definition}
יהא%
\newlecture{25 באוקטובר}{2018}
$R$
חוג ויהיו
$a,b\in \R \setminus \set{0}$.
$d$
נקרא
\stress{מחלק משותף גדול ביותר של $a$%
ו־%
$b$}
אם מתקיימות התכונות הבאות.
\begin{enumerate}
\item $d \mid a,b$
\item אם
$d' \in R$
מקיים
$d' \mid a,b$
אז
$d' \mid d$.
\end{enumerate}
\end{definition}

\begin{proposition}
יהא
$R$
חוג אוקלידי ויהיו
$a,b \in R$
שונים מאפס אז קיים מחלק משותף גדול ביותר ל־%
$a$
ו־%
$b$.
\end{proposition}
\begin{proof}
יהיו
$a,b \in R \setminus \set{0}$
ויהא
$I = \trs{a,b}$
האידאל הנוצר על ידי
$a$
ו-%
$b$.
לפי הטענה, יש ל־%
$I$
יוצר
$d$,
ונראה כי זהו ממג"ב (מחלק משותף גדול ביותר) של
$a,b$.
\begin{description}
\item[מחלק משותף:]
ניתן לכתוב
$a = 1\cdot a \in I$
לכן
$d \mid b$.
גם
$b = 1\cdot b \in I$
לכן
$d \mid b$.
\item[מקסימליות:]
אם
$d' \mid b$
וגם
$d' \mid b$.
קיימים
$x_1, y_1 \in R$
עבורם
$d = x_1a + y_1b$.
כעת
$d' \mid a,b$
ולכן
$d' \mid d$.
\end{description}
\end{proof}
\begin{definition}
איברים
$a,b \in R$
נקראים
\stress{חברים}
אם קיים איבר הפיך
$u \in R^{\times}$
עבורו
$a = bu$.
\end{definition}
\begin{note}
חברות זה יחס שקילות.
\end{note}
\begin{proposition}
יהיו
$a,b \in \R \setminus \set{0}$
עם
$d,d'$
ממג"ב. אז
$d,d'$
חברים.
\end{proposition}
\begin{proof}
מהגדרת ממג"ב מתקיים
$d' \mid d$
וגם
$d \mid d'$.
לכן יש
$x,y$
עבורם
$d = xd'$
וגם
$d' = yd$.
נציב את השיוויון השני בראשון ונקבל
$d = xyd$
לכן
$xy = 1$
ונקבל כי
$x,y$
הפיכים.
לכן
$d,d'$
חברים.
\end{proof}
\begin{corollary}
יהא
$d$
ממג"ב של
$a,b \in \R$.
קיימים
$x,y \in R$
כך שמתקיים
$d = xa + yb$.
\end{corollary}
\begin{corollary}
נמצא ממג"ב אחד
$d'$
עבורו
$\prs{d} = \trs{a,b}$.
$d,d'$
חברים ולכן יוצרים את אותו האידאל
$\prs{d'} = \prs{d}$.
לכן גם
$d'$
צירוף לינארי של
$a,b$
עם מקדמים ב־%
$R$.
\end{corollary}
\begin{example}[חוג השלמים הגאוסים]
נגדיר
$\ZZ\brs{i} = \set{a+bi}{a,b \in \ZZ} \subseteq \CC$.
\end{example}
\begin{proposition}
$\ZZ\brs{i}$
חוג אוקלידי.
\end{proposition}
\begin{proof}
נזכיר כי בשלמים יש חלוקה עם שארית
$b = qa + r$
עם התנאי
$\abs{r} \leq \frac{\abs{a}}{2}$.
נגדיר
$N\prs{a+bi} = a^2 + b^2 = \abs{a+bi}^2$.
יהיו
$a+bi, c+di \in R$.
נעשה חלוקה עם שארית ל־%
$a+bi, c+di$.
מתקיים קודם כל
\begin{equation}\label{gaussian}
\text{.} \frac{a+bi}{c+di} = \frac{ac + bd}{c^2 + d^2} + \frac{bc - ad}{c^2 + d^2}i
\end{equation}
נחפש מספר ב־%
$\ZZ[i]$
קרוב ביותר למנה זאת.
נעשה חלוקה עם שארית ב־%
$\ZZ$
במקום המקדמים במנה.
\begin{align*}
ac + bd &= x_1\prs{c^2 + d^2} + r_1 \\
bc - ad &= x_1\prs{c^2 + d^2}+r_2
\end{align*}
כאשר
$\abs{r_i} \leq \frac{c^2 + d^2}{2}$.
נציב בנוסחה
\ref{gaussian}
ונקבל
\begin{align*}
\frac{a+bi}{c+di} &= \frac{x_1\prs{c^2 + d^2} + r_1 + \prs{x_2\prs{c^2 + d^2} + r_2}i}{c^2 + d^2}\\
&= x_1 + x_2i + \frac{r_1 + r_2 i}{c^2 + d^2}
\end{align*}
או לאחר כפל שני האגפים
\begin{align*}
\text{.} a+bi = \prs{x_1 + x_2i}\prs{c+di}+\frac{r_1 + r_2i}{c^2 + d^2}\prs{c + di}
\end{align*}
נטען כי זאת חלוקה עם שארית. יש להראות כי הביטוי
$\frac{r_1 + r_2 i}{c^2+d^2}\prs{c+di}$
שלם גאוסי וכי הנורמה שלו קטנה מזאת של
$c+di$.
אכן זהו שלם גאוסי כיוון שניתן לכתוב
\[\text{.} \frac{r_1 + r_2i}{c^2 + d^2} \prs{c+di} = a+bi - \prs{x_1 + x_2i}\prs{c+di} \in \ZZ[i]\]
נשאיר את סיום ההוכחה כתרגיל.
\end{proof}
\begin{exercise}
הוכיחו את אי־השיוויון הבא כדי לסיים את ההוכחה.
\[\abs{\frac{\prs{r_1 + r_2}\prs{c+di}}{c^2 + d^2}}^2 < \abs{c+di}^2\]
\end{exercise}
\section{האלגוריתם של אוקלידס}
יהא%
\newtutorial{25 באוקטובר}{2018}
$R$
חוג אוקלידי ויהיו
$a,b \in \R \setminus \set{0}$.
האלגוריתם של אוקלידס מוצא ממג"ב של
$a$
ו־%
$b$.

\begin{algorithm}
\begin{enumerate}
\item נסמן
$b = r_0$.
\item נכתוב
$a = q_1 b + r_1$.
\item נחלק את
$r_{i-1}$
ב־%
$r_{i}$
עם
$i$
מקסימלי.
נכתוב
$r_{i-1} = q_{i+1}r_i + r_{i+1}$.
\\ נפסיק כשנקבל
$r_{n+1} = 0$
ואז
$r_n$
הוא ממג"ב של
$a,b$.
\end{enumerate}
\end{algorithm}
\begin{exercise}
מצאו ממג"ב של
$91$
ו־%
$35$.
\end{exercise}
\begin{solution}
\begin{align*}
91 &= 2\cdot 35 + 21 \\
35 &= 1 \cdot 21 + 14 \\
21 &= 1\cdot 14 + 7 \\
14 &= 2\cdot 7 + 0
\end{align*}
לכן
$\gcd\prs{91,35} = 7$.
\end{solution}
\begin{exercise}
מצאו את
$\gcd\prs{13 + 13i, -1+18i}$.
\end{exercise}
\begin{solution}
נציג שני פתרונות.
\begin{enumerate}
\item נבצע חלוקה עם שארית.
מתקיים
\begin{align} \label{gaussian_exercise}
\text{.}\frac{13 + 13i}{-1+18i} = \frac{17}{25}-\frac{19}{25}i
\end{align}
נבצע חלוקה עם שארית בשלמים.
\begin{align*}
17 &= 1\cdot 25 + \prs{-8} \\
-19 &= -1\cdot 25 + 6
\end{align*}
נציב ב־%
\ref{gaussian_exercise}
ונקבל
\begin{align*}
\text{.} \frac{13 + 13i}{-1+18i} &= \frac{28 - 8 - 25i + 6i}{25} = 1-i + \frac{-8 + 6i}{25}
\end{align*}
נכפול ונקבל
\begin{align*}
13+13i &= \prs{1-i}\prs{-1+18i} + \frac{-8 + 6i}{25}\prs{-1+18i} \\
\text{.} \phantom{13+13i} &= \prs{1-i}\prs{-1+18i} + \prs{-4 -6i}
\end{align*}
כעת נחלק את
$\prs{-1 + 18i}$
בשארית
$-4-6i$.
יוצא
\[\text{.} -1 + 18i = \prs{-2-2i}\prs{-4-6i} + 3 - 2i\]
מחלקים שוב
$-4+6i = \prs{-2i}\prs{3-2i} + 0$
ולכן
$\gcd\prs{13+13i, -1+18i} = 3-2i$.
\item
נזכיר טענה.
\begin{proposition}
יהא
$a+bi \in \ZZ[i]$.
אם
$N\prs{a+bi} = a^2 + b^2$
ראשוני ב־%
$\NN$
אז
$a+bi$
ראשוני ב־%
$\ZZ[i]$.
\end{proposition}
נפרק את
$13 + 13i$
ואת
$-1+18i$
למכפלות ראשוניים.
מתקיים
$13 + 13i = 13\prs{1+i}$
כאשר
$N\prs{1+i} = 1^2 + 1^2 = 2$
ראשוני, לכן
$1+i$
ראשוני.
ניתן לכתוב
$13 = \prs{2+3i}\prs{2-3i}$
כאשר מהטענה זה פירוק לראשוניים.
לכן
\[13+13i = \prs{2+3i}\prs{2-3i}\prs{1+i}\]
פירוק לראשוניים.
\\
נפרק את
$-1 + 18i$.
מתקיים
\[\text{.} N\prs{-1+18i} = 1^2 + 18^2 = 325 = 5^2 \cdot 13\]
הנורמה אצלנו כפלית ולכן למחלקים נורמות בקבוצה
$\set{5, 5^2, 13}$
(נפרט יותר בהרצאה).
נחלק את
$-1 + 18i$
ב־%
$2+3i$.
יוצא
\[\prs{-1+18i} = \prs{2+3i}\prs{1+2i}\prs{2-i}\]
נקבל כי $2+3i$
הוא הגורם המשותף היחיד בפירוק לראשוניים עד־כדי חברות (לחברים יש אותה הנורמה) ולכן
$\gcd\prs{13+13i, -1+18i} = 2+3i$.
\end{enumerate}
\end{solution}
\begin{theorem}[אוקלידס]
יש אינסוף ראשוניים ב־%
$\NN$.
\end{theorem}
\begin{proof}
נניח בשלילה שיש מספר סופי של ראשוניים
$p_1, \ldots, p_k$
ונסמן
$N = \prs{\prod_{i=1}^k p_i} + 1$.
אם
$p_i \in N$
אז
$p_i \mid 1$
וזו סתירה לכך שיש ראשוני שמחלק את
$N$.
\end{proof}
\begin{exercise}
יש
ב־%
$\NN$
אינסוף ראשוניים
$p$
שמקיימים
$p \equiv 3 \mod{4}$.
\end{exercise}
\begin{solution}
נניח שיש מספר סופי של ראשוניים
$p_1, \ldots, p_k \equiv 3 \mod{4}$.
ניקח
$N = 4 \prs{\prod_{i=1}^k p_i} - 1$
ואז
$p_i \nmid N$
לכל
$i$.
נפרק את
$N$
לראשוניים
$N = \prod_{i=1}^m q_i$.
אז קיים
$q_i \equiv 3 \mod{4}$
כי אחרת
\[ N \equiv \prod_{i=1}^m q_i \equiv \prod_{i=1}^m 1 \equiv 1 \mod{4}\]
בסתירה.
אבל
$q_i \neq p_j$
לכל
$j \in [k]$,
בסתירה.
\end{solution}
\backmatter
\end{document}

