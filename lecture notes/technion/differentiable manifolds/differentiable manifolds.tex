\documentclass[a4paper,10pt,twoside,openany]{book}

\usepackage[lang=hebrew]{maths}
\usepackage{hebrewdoc}
\usepackage{stylish}
\usepackage{lipsum}

\title{סיכומי הרצאות ביריעות דיפרנציאביליות \\ \large{חורף 2018, הטכניון}}
\author{הרצאותיו של מיכאל חנבסקי \\ \large סוכמו על ידי אלעד צורני}
\date{\today}

\begin{document}
\frontmatter
\frontpage{atlas}{0.8\textwidth}{מפת העולם. על ידי ג'ררד ואן שגן.}
\tableofcontents
\countlectures
\newpage

\chapter*{הקדמה}
\addcontentsline{toc}{chapter}{הקדמה} \markboth{הקדמה}{}

\section*{הבהרה}
\addcontentsline{toc}{section}{הבהרה} %\markboth{Technicalities}{}

סיכומי הרצאות אלו אינם רשמיים ולכן אין
\emph{כל הבטחה}
כי החומר המוקלד הינו בהתאמה כלשהי עם דרישות הקורס, או שהינו חסר טעויות.
\\
להיפך, ודאי ישנן טעויות בסיכום! אעריך אם הערות ותיקונים ישלחו אלי בכתובת דוא"ל
\textenglish{\href{mailto:tzorani.elad@gmail.com}{tzorani.elad@gmail.com}}.\\
אלעד צורני.

\section*{ספרות מומלצת.}
\addcontentsline{toc}{section}{ספרות מומלצת} %\markboth{Course Literature}{}

הספרות המומלצת עבור הקורס הינה כדלהלן.

\begin{english}
\begin{description}
\item[M. Spivak:] Calculus on manifolds: a modern approach to classical theorems of advanced calculus

\item[J. Milnor:] Topology from athe differentiable viewpoint

\item[J. Lee:] Introduction to smooth manifolds

\item[L. Conlon:] Differentiable manifolds: a first course

\item[V. Guillemin, A, Pollcak:] Differential topology
\end{description}
\end{english}

\mainmatter

\chapter*{הקדמה}
\section{תוכן הקורס}
\stress{יריעה} \newlecture{28 באוקטובר}%
{2018}
היא מרחב טופולוגי שלוקלית נראה כמו קבוצה פתוחה ב־%
$\RR^n$.%
\newlecture{21 באוקטובר}{2018}
דוגמאות ליריעות הן עקומות ומשטחים.
\begin{example}
$S^2$
הינה יריעה. סביבה של נקודה נראית כמו קבוצה פתוחה ב־%
$\RR^2$.
\end{example}
\begin{example}
$\TT^2$
הינה יריעה. סביבה של נקודה גם כאן נראית כמו קבוצה פתוחה ב־%
$\RR^2$.
\end{example}
נסתכל על העתקה
$f \colon S^2 \to \TT^2$.
נוכל לזהות את ההעתקה
$f \colon \RR^2 \to \RR^2$
באופן לוקלי עם העתקה
$\hat{f}$
שמעתיקה קבוצות פתוחות לקבוצות פתוחות. ראו איור
\ref{fig1}.
\begin{figure}[ht]
\centering
\caption{העתקה לוקלית.}
\label{fig1}
\includegraphics[width=0.8\textwidth]{sources/figure1}
\end{figure}
\\
בקורס נעסוק בהכללת משפטים מאנליזה ב־%
$\RR^n$
למשפטים על יריעות חלקות.
רוב המשפטים ניתנים להכללה ליריעות $\CCC^k$.
למשפטים מהקורס שימושים רבים בגיאומטריה, טופולוגיה, פיזיקה, אנליזה, קומבינטוריקה ואלגברה. כדי שהתיאוריה הקשורה לקורס לא תישאר באוויר נראה חלק גדול מהקורס שימושים, בעיקר בתחום של טופולוגיה דיפרנציאלית.
\\
קיימים מספר משפטים מוכרים מאנליזה.
\\
\begin{examples}
\begin{enumerate}
\item \emph{נוסחאת ניוטון-לייבניץ:}
\[\int_{\del\brs{a,b}} F = \int_{\brs{a,b}} F' \diff x\]
\item \emph{נוסחאת גרין:}
\[\int_{\del U} f\diff x + g \diff y = \iint_{U} \prs{\frac{\del g}{\del x} - \frac{\del f}{\del{y}}} \diff x \diff y\]
\item \emph{משפט גאוס:}
\[\iiint_{\Omega} \div \vec{F} \diff v = \iint_{\del \Omega} \trs{\vec{F},\vec{n}} \diff s\]
\item \emph{משפט סטוקס:}
\[\iint_{\Sigma} \trs{\rot \vec{F}, \vec{n}} \diff s = \int_{\del \Sigma} \trs{\vec{F}, \vec{\gamma}} \diff t\]
\end{enumerate}
\end{examples}
כולם נובעים ממשפט סטוקס כללי יותר אותו נוכיח בקורס.
\[\int_{\del \Omega} \omega = \int_{\Omega} \diff \omega\]

\section{דרישות קדם}
נשתמש בקורס רבות במשפט הפונקציה הסתמונה ובמשפט הפונקציה ההפוכה מ־%
$\RR^n$.
רצוי להכיר את ניסוחיהם, את המשמעויות הגיאומטריות ומספר שימושים שלהם.
נשתמש בהחלפות קואורדינטות מאלגברה לינאריות, במשפט פוביני, מטריצות יעקוביאן, במטריצות יעקוביאן, בכלל השרשרת ובמשפט קיום ויחידות של מד"ר.

\section{תרגילי בית}
בקורס יפורסמו ארבעה תרגילי בית רשמיים, רובם ברמת הבנת ההגדרות. אין חובת הגשה אך מומלץ מאוד לפתור את התרגילים כדי לוודא שאינכם הולכים לאיבוד.

\section{ציון}
הציון הסופי כולו יסתמך על הגשת עבודת בית.

\chapter{מבוא}
\section{הגדרות}
\begin{definition}
מרחב טופולוגי
$M$
נקרא
\stress{יריעה טופולוגית}
\textenglish{(topological manifold)}
אם לכל
$x \in M$
יש סביבה
$U$
שהומאומורפית לקבוצה פתוחה ב־%
$\RR^n$
עבור
$n$
כלשהו.
\end{definition}
\begin{remark}
לעתים יריעה טופולוגית כפי שהגדרנו אותה נקראת
\stress{מרחב אוקלידית לוקלית}
\textenglish{(locally Euclidean space)}.
\end{remark}
\begin{fact}
עבור יריעה טופולוגית,
$n$
קבוע מקומית.
\\
אם
$M$
קשירה,
$n$
קבוע.
\end{fact}
\begin{definition}
עבור יריעה
$M$
עם
$n$
קבוע, נגיד ש־%
$n$
הוא
\stress{המימד}
של היריעה.
\end{definition}
\begin{exercise}
מצאו אילו אותיות מבין
\textenglish{MANIFOLD}
הן יריעות. מצאו אילו הומאומורפיות.
\end{exercise}
\begin{examples}
\begin{enumerate}
\item עקומות
\item משטח ב־%
$\RR^3$
\item $S^n \subseteq \RR^{n+1}$:
נסמן
$\Uu^+ = S^n \setminus \set{p^+}$
נתאים לנקודה
$x \in \Uu^+$
על הספירה את הנקודה
$\phi^{+}(x) \in \RR^n$
המתקבלת על ידי הטלה סטרוגרפית. ראו איור
\ref{fig2}.
\begin{figure}[ht]
\centering
\caption{הטלה סטרוגרפית.}
\label{fig2}
\includegraphics[width=0.8\textwidth]{sources/figure2}
\end{figure}
ניתן לראות כי
$\phi^+$
הומאומורפיזם.
אפשר באותו אופן להגדיר
$\phi^{-} \colon \U^{-} \to \RR^n$.
לכל נקודה על
$S^n$
יש סביבה הומאומורפית לקבוצה פתוחה ב־%
$\RR^n$
המתקבלת מהמקור דרך
$\phi^{\pm}$
של סביבה בתמונה, לכן
$S^n$
יריעה.
\end{enumerate}
\end{examples}
\begin{exercise}
אם
$M_1, M_2$
יריעות טופולוגיות אז
$M_1 \times M_2$
עם טופולוגיית המכפלה הינה יריעה.
אם המימד של
$M_1, M_2$
אחיד מתקיים גם
\[\text{.} \dim\prs{M_1 \times M_2} = \dim{M_1} + \dim{M_2}\]
\end{exercise}
\begin{example}
\stress{טורוס
$n$–%
מימדי} הוא
$\TT^n \ceq \prod_{k=1}^n S^1$.
ניתן להגדיר גם
$T^n = \quot{\RR^n}{\ZZ^n} = \quot{\RR^n}{\sim}$
כאשר
$x \sim y$
אם
$x-y \in \ZZ^n$.
נסמן
$\pi \colon \RR^n \to T^n$
את ההטלה הטבעית (כלומר
$\pi\prs{x} = \brs{x} = x+\ZZ^n$)
ואז
$\Uu \subseteq T^n$
פתוחה אם
$\pi^{-1}\prs{\Uu}$
פתוחה ב־%
$\RR^n$.
\end{example}
\begin{exercise}
הראו כי
$\TT^n$
הומאומורפי ל־%
$T^n$.
\end{exercise}
\begin{example}
נגדיר
\stress{מרחב פרויקטיבי הוא}
בשתי דרכים.
\begin{enumerate}[label = (\roman*)]
\item
נסתכל על ישרים דרך
$\vec{0}$
ב־%
$\RR^{n-1}$.
עבור
$\ell$
ישר דרך
$0$
נסמן
$\Uu_{\eps}\prs{\ell}$
את אוסף הישרים
$\ell'$
דרך
$\vec{0}$
כך שמתקיים
$\deg{\ell, \ell'} < \eps$.
אז קבוצה פתוחה היא איחוד כלשהו של
$\Uu_{i\in I} \Uu_{\eps_i}\prs{\ell_i}$.
\item
נגדיר גם
$\mrm{RP}^n = \quot{S^n}{\pm 1}$
כלומר
$x \sim y$
אם
$x = \pm y$.
נסמן
$\pi \colon S^n \to \mrm{RP}^n$
את ההטלה
$x \to \set{x, -x}$
ואז
$\Uu \subseteq \mrm{RP}^n$
פתוחה אם
$\pi^{-1}\prs{\Uu}$
פתוחה ב־%
$\S^n$.
\end{enumerate}
\end{example}
\begin{exercise}
הראו כי
$\mrm{RP}^n$
הומאומורפי ל־%
$\mrm{RP}^n$.
\end{exercise}
\begin{example}
$\R \mrm{P}^1$
הומאומורפי ל־%
$\S^1$.
לאחר הזיהוי מתקבל קטע המזוהה בקצותיו, וזה הומאומורפי למעגל.
\end{example}
\begin{definition}
תהי
$M$
יריעה טופולוגית.
\stress{מפה}
\textenglish{map / coordinate chart}
היא זוג
$\prs{\Uu, \phi}$
כאשר
$\Uu \subseteq M$
קבוצה פתוחה ו־%
$\phi \colon \Uu \to \phi(\Uu) \subseteq \RR^n$
הומאומורפיזם, ו־%
$\phi\prs{\Uu} \subseteq \RR^n$
פתוחה.
\end{definition}
\begin{example}
ב־%
$S^n$
הגדרנו שתי מפות
$\prs{\Uu^{\pm}, \phi^{\pm}}$.
\end{example}
\begin{definition}
תהיינה
$\prs{\Uu, \phi}, \prs{\Vv, \psi}$
מפות עבור יריעה
$M$.
אז
$\phi \circ \psi^{-1} \colon \psi\prs{\Uu \cap \Vv} \to \phi\prs{\Uu \cap \Vv}$
הומאומורפיזם שנקרא
\stress{פונקציית מעבר} \textenglish{transition map}
או
\stress{פונקציית החלפת קואורדינטות}.
ראו איור
\ref{fig3}
\begin{figure}[ht]
\centering
\caption{פונקציית מעבר.}
\label{fig3}
\includegraphics[width=0.8\textwidth]{sources/figure3}
\end{figure}
\end{definition}
\begin{definition}
תהי העתקה
$f \colon M \to N$
בין יריעות, ונניח בה"כ
$f\prs{\Uu_1} \subseteq \Vv_1$.
אז
$\hat{f}_1 \colon \phi_1\prs{\Uu_1} \to \psi_1\prs{\Vv_1}$
\stress{הצגה מקומית של
$f$}
(ביחס למפות
$\prs{\phi_1, \Uu_1}$
ו־%
$\prs{\psi_1, \Vv_1}$).
מתקיים
$\hat{f}_1 = \eval{\psi_1 \circ f \circ \phi_1^{-1}}{\psi_1\prs{\Uu_1}}{}$.
ראו איור
\ref{fig4}.
\begin{figure}[ht]
\centering
\caption{הצגה מקומית.}
\label{fig4}
\includegraphics[width=0.8\textwidth]{sources/figure4}
\end{figure}
\end{definition}
\begin{definition}
תהיינה
$\hat{f}_1, \hat{f}_2$
שתי הצגות מקומיות של
$f \colon M \to N$.
אז
\[\text{.} \eval{\hat{f}_2}{\phi_2\prs{\Uu_1 \cap \Uu_2}}{} = \psi_2 \circ \psi_1^{-1} \circ \hat{f}_1 \circ \phi_1 \circ \phi_2^{-1}\]
כאן
$\phi_1 \circ \phi_2^{-1}$
פונקציית מעבר מ־%
$\prs{\Uu_2, \psi_2}$
ל־%
$\prs{\Uu_1, \phi_1}$
ו־%
$\psi_2 \circ \psi_1^{-1}$
פונקציית מעבר מ־%
$\prs{\Vv, \psi_1}$
ל־%
$\prs{\Vv, \psi_2}$.
ראו איור
\ref{fig5}.
\begin{figure}[ht]
\centering
\caption{פונקציית מעבר בין יריעות.}
\label{fig5}
\includegraphics[width=0.8\textwidth]{sources/figure5}
\end{figure}
\end{definition}
נזכיר כי
$f \colon \Uu \to \RR^m$
עבור
$\Uu \subseteq \RR^n$
נקראת
\stress{חלקה}
\textenglish{(smooth)}
אם לכל
$x \in \Uu$
קיימות ורציפות נגזרות חלקיות של
$f$
מסדר כלשהו.
\begin{notation}
עבור
$f$
חלקה נסמן
$f \in \CCC^{\infty}$.
\end{notation}
\begin{definition}
$f \colon \Uu \to \Ww$
עבור
$\Uu \subseteq \RR^n, \Ww \subseteq \RR^m$
\stress{דיפאומורפיזם}
אם
$f$
הפיכה, ו־%
$f,f^{-1}$
חלקות.
\end{definition}
\begin{exercise}
אם
$f \colon \Uu \to \Ww$
חלקה וגם
$\Uu \subseteq \RR^n, \Ww \subseteq \RR^m$
אז
$m=n$.
\end{exercise}
\begin{examples}
\item $f(x) \ceq \fcases{0 & x < 0 \\ x^2 & x \geq 0}$
איננה חלקה.
\item $f(x) \ceq \fcases{0 & x \leq 0 \\ e^{-\frac{1}{x}} & x > 0}$
חלקה (אך איננה אנליטית).
\item נגדיר $\Uu = \prs{-\frac{\pi}{2}, \frac{\pi}{2}}, \Ww = \R$.
אז
$\tan \colon \Uu \to \Ww$
דיפאומורפיזם.
\end{examples}
\begin{exercise}
\begin{enumerate}
\item אם
$F$
דיפאו' גם
$F^{-1}$
דיפאו'.
\item הרכבה של דיפאו' היא דיפאו'.
\item אם
$F_1 \colon \Uu_1 \to \Ww_1$
ו־%
$F_2 \colon \Uu_2 \to \Ww_2$
דיפאו' אז
$F_1 \times F_2 \colon \Uu_1 \times \Uu_2 \to \Ww_1 \to \Ww_2$
דיפאו'.
\item אם
$-\infty \leq a \leq b \leq \infty$
ו־%
$-\infty \leq c \leq d \leq \infty$
אז
$\prs{a,b}$
דיפאומורפי ל־%
$\prs{c,d}$.
\item הקבוצות במישור
מאיור
\ref{fig6}
דיפאומורפיות.
\begin{figure}[ht]
\centering
\caption{קבוצות דיפאומורפיות במישור.}
\label{fig6}
\includegraphics[width=0.8\textwidth]{sources/figure6}
\end{figure}
\end{enumerate}
\end{exercise}
\section{מבנה חלק}
נרצה להגדיר מתי העתקה
$f \colon M \to \RR$
הינה גזירה/חלקה בנקודה
$p$.
ננסה להגדיר גזירות של העתקה
$f$
כנ"ל.
\begin{description}
\item["הגדרה":]
\emph{$f$
גזירה ב־%
$p \in M$
אם כל הצגה מקומית
$\hat{f}$
גזירה ב־%
$\phi\prs{p}$.}
הגדרה זאת איננה טובה כי בהינתן שתי הצגות מקומיות סביב
$p$
ופונקציית מעבר
$\psi$,
$\hat{f}_1 = \hat{f}_2 \circ \psi$
איננה בהכרח גזירה.

\item["הגדרה 2":]
נניח כי
$M \subseteq \RR^n$
יריעה. נאמר כי
$f \colon M \to \RR$
חלקה אם ניתן להרחיב את
$f$
לפונקציה חלקה
$f \colon \Uu \to \RR$
כאשר
$\Uu$
סביבה פתוחה של
$M$.
\\
גם בהגדרה זו יש בעיות, כפי שנראה מיד בדוגמה.
\end{description}
\begin{example}
ניקח
$M = \RR \subseteq \RR$.
נסתכל על שלוש מפות
$\prs{\Uu_i, \phi_i}, i \in [3]$
כאשר
$\Uu_i = \RR$
והמפות מוגדרות על ידי
\begin{align*}
\phi_1(x) &= x \\
\phi_2(x) &= x^3 \\
\text{.} \phi_3(x) &= \sqrt[3]{x}
\end{align*}
תהי
$f(x) = \sum_{i=0}^{\infty} a_i x^i$.
אז
$\hat{f}_1 = f$
ונקבל כי מה"הגדרה"
$\hat{f}_1$
חלקה אם ורק אם
$f$
חלקה במובן של אינפי.
נקבל
\[\hat{f}_2(u) = f\circ \phi_2^{-1}(u) = \sum_{i=0}^{\infty} a_i \cdot u^{\frac{i}{3}}\]
לכן תנאי הכרחי עבור חלקות של
$\hat{f}_2$
הוא ש־%
$a_i = 0$
לכל
$i \not\equiv 0 \mod{3}$.
נקבל באותו אופן
\[\hat{f}_3\prs{w} = \sum_{i=1}^{\infty} a_i w^{3i}\]
ונקבל פחות הצגות חלקות מאשר בהצגה המקומית הראשונה.
\end{example}
\begin{example}
נגדיר
$M_1 = \set{\prs{x,y} \in \RR^2}{y=0}$
ו־%
$M_2 = \set{\prs{x,y} \in \RR^2}{y=\abs{x}}$.
נגדיר העתקה
$P \colon M_2 \to M_1$
ע"י
$\prs{x,\abs{x}} \mapsto \prs{x,0}$.
זהו הומיאומורפיזם.
\begin{exercise}
$f_1 \colon M_1 \to \RR$
ע"י
$\prs{x,0} \mapsto \phi\prs{x}$
חלקה אם ורק אם
$\phi$
חלקה כפונקציה
$\RR \to \RR$.
\end{exercise}
\begin{exercise}
$f_2\prs{x, \abs{x}} = \abs{x}$
פונקצייה חלקה
$M_2 \to \RR$.\\
רמז: ניתן להרחיב אותה לפונקציה
$f_2\prs{x,y} = y$
כאשר
$f_2 \colon \RR^2 \to \RR$.
\end{exercise}
כאשר נפעיל את הזיהוי בין
$M_1$
ל־%
$M_2$
על ידי
$P$
נקבל כי
$f_2$
פונקצייה חלקה!
לכן ה"הגדרה" תלויה בשיכון
$M \rmono \RR^n$
ולא רק ביריעה עצמה.
\end{example}
נתקלנו בבעייה. המבנה של יריעה טופולוגית הינו חלש מידי ונראה שאין סיכוי להגדיר עליו נגזרת. איך נעקוף זאת?
נרצה לדרוש שההעתקות המעבר יהיו חלקות, ואז
$\hat{f}_1 = \psi \circ \hat{f}_2 \circ \phi$
תהיה גזירה אם ורק אם
$\hat{f}_2$
גזירה, עבור
$\psi, \phi$
העתקות מעבר.\\
\begin{definition}
תהי
$M$
יריעה. מפות
$\prs{\Uu, \phi}, \prs{\Vv, \psi}$
יקראו
\stress{מתואמות}
(\textenglish{compatible})
אם
$\psi \circ \phi^{-1}, \phi \circ \psi^{-1}$
חלקות (כלומר, דיפאומורפיזמים).
\end{definition}
\begin{exercise}
תהי
$f \colon \Uu \cap \Vv \to \RR$.
אז
$f \circ \phi^{-1}$
חלקה אם ורק אם
$f \circ \psi^{-1}$
חלקה, כאשר
$\phi,\psi$
מתואמות.
\end{exercise}
\begin{definition}
תהי
$M$
יריעה.
\stress{אטלס חלק}
\textenglish{(smooth atlas / $\CCC^{\infty}$ atlas)}
על
$M$
הוא משפחת מפות מתואמות
$\set{\prs{\Uu_{\alpha}, \phi_{\alpha}}}_{\alpha \in I}$
כך שמתקיים
$M = \bigcup_{\alpha \in I} \Uu_{\alpha}$.
\end{definition}
\begin{example}
ניקח
$M = S^1$
ואז
$\set{\prs{\Uu^{\pm}, \phi^{\pm}}}$
שהגדרנו קודם הינו אטלס.
\end{example}
\begin{claim}
האטלס הנ"ל הינו חלק.
\end{claim}
\begin{proof}
\begin{figure}[h!]
\caption{the stereographic map.}
\label{stereographic}
\includegraphics[scale=1]{sources/stereographic}
\end{figure}
מתקיים
\[\text{.} \Uu^+ \cap \Uu^- = S^1 \setminus \set{p^+, p^-}\]
כמובן
$\phi^{\pm}\prs{\Uu^+ \cap \Uu^-} = \RR \setminus \set{0}$.
ראו איור
\ref{stereographic}.
מתקיים
$\phi^-\prs{Q} = \tan\prs{\beta}$
וגם
$\phi^+\prs{Q} = \frac{1}{\tan\prs{\beta}}$.
לכן אם
$x = \phi^+\prs{Q}$
נקבל
$\phi^- \circ \prs{\phi^+}^{-1}\prs{x} = \frac{1}{x}$
פונקציית מעבר חלקה.
כנ"ל לפונקציית המעבר לכיוון השני, לכן האטלס חלק.
\end{proof}
\begin{definition}
שני אטלסים חלקים נקראים
\stress{שקולים}
אם האיחוד שלהם הוא אטלס חלק.
\end{definition}
\begin{exercise}
שקילות זאת הינה אכן יחס שקילות בין אטלסים חלקים על
$M$.
\end{exercise}
\begin{definition}
\stress{מבנה חלק}
על
$M$
זו מחלקת שקילות של אטלסים.
\end{definition}
\begin{example}
ניקח
$M = \RR$
עם
$\Uu_i = \RR$
וההעתקות
$\phi_1 = \id, \phi_2 = x^2, \phi_3 = \sqrt[3]{x}$.
אז שלושת האטלסים
$\set{\prs{\Uu_i, \phi_i}}$
אטלסים לא מתואמים בזוגות ולכן מייצגים שלושה מבנים חלקים שונים.
\end{example}
\begin{definition}
תהי
$\prs{M,\mcal{A}}$
יריעה עם אטלס חלק.
נגדיר
$f \colon M \to \RR$
\stress{גזירה}
בנקודה
$p \in M$
אם
$\hat{f} \colon \phi\prs{\Uu} \to \RR$
גזירה ב־%
$\phi\prs{p}$
עבור
$\prs{\Uu, \phi}$
מפה סביבה
$p$
מהאטלס
$\mcal{A}$.
\end{definition}
\begin{exercise}
בדקו כי הנ"ל מוגדר היטב.
\end{exercise}
\begin{definition}
\stress{יריעה חלקה}
היא יריעה טופולוגית
$M$
עם בחירה של אטלס חלק על
$M$.
\end{definition}
\begin{claim}
בכל מחלקת שקילות של אטלסים קיים אטלס מקסימלי.
\end{claim}
\begin{corollary}
בחירת מבנה חלק שקולה לבחירת אטלס חלק מקסימלי על
$M$.
\end{corollary}
\begin{definition}
תהיינה
$\prs{N, \Aa_N}, \prs{M, \Aa_M}$
שתי יריעות חלקות ותהי
$f \colon M \to N$.
אז
$f$
תיקרא
\stress{גזירה}
ב־%
$p \in M$
אם
$\hat{f} = \psi \circ f \circ \phi^{-1}$
גזירה ב־%
$\phi\prs{p}$,
ההצגה המקומית ביחס למפות
$\prs{\Uu, \phi} \in \Aa_M, p \in \Uu$
ו־%
$\prs{\Vv, \psi} \in \Aa_N, f\prs{p} \in \Vv$.
\end{definition}
\begin{exercise}
בדקו כי הנ"ל מוגדר היטב ביחס למפות מתואמות.
\end{exercise}
\begin{definition}
פונקציה
$f \colon M \to N$
\stress{חלקה}
אם כל ההצגות המקומיות של
$f$
ביחס למפות מתואמות הן פונקציות חלקות.%
\newlecture{28 באוקטובר}%
{2018}
\end{definition}
\begin{remark}
כדי לבדוק חלקות, די לבדוק חלקות של פונקציות מתואמות עם מפות שמכסות את היריעה.
\end{remark}
\begin{definition}
פונקציה הפיכה
$f$
כאשר
$f,f^{-1}$
חלקות נקראת
\stress{דיפאומורפיזם}.
\end{definition}
\begin{exercise}
אם
$f \colon M \to N$
דיפאומורפיזם, אז
$\dim M = \dim N$.
\end{exercise}
\begin{exercise}
דיפאומורפיזמים שומרים על פונקציות חלקות.
אם בדיאגרמה הבאה
$\phi,\psi$
איזומורפיזמים, אז
$h$
חלקה אם ורק אם
$h'$
חלקה.
\[
\begin{tikzcd}
M \arrow[r,"h"] \arrow[d,"\phi"] & N \arrow[d,"\psi"] \\
M' \arrow[r, dotted, "h'"] & N'
\end{tikzcd}
\]
\end{exercise}
\begin{definition}
\stress{יריעה דיפרנציאבילית
$\CCC^r$}
היא יריעה טופולוגית עם אטלס בו פונקציות המעבר חלקות
$\CCC^r$.
\end{definition}
\begin{exercise}
תהי
$\prs{M,\Aa}$
יריעה חלקה ותהי
$\prs{\Uu, \phi}$
מפה מתואמת עם
$\Aa$.
אז
$\phi \colon \Uu \to \phi\prs{\Uu} \subseteq \RR^m$
דיפאומורפיזם.
\end{exercise}
\begin{example}
תהי
$M = \RR$
עם
$\Aa_1 = \set{\prs{\RR,\1}}$
המבנה החלק הסטנדרטי, ועם
$\Aa_2 = \set{\prs{\RR,x^3}}$.
אז
$\prs{M,\Aa_1}, \prs{M,\Aa_2}$
יריעות חלקות שונות, אך דיפאומורפיות.
נסמן
$f(x) = \sqrt[3]{x}$.
אז
$\hat{f} = \1_{\RR}$
ולכן
$f$
דיפאומורפיזם.
\begin{figure}[h!]
\centering
\caption{Diffeomorphism.}
\label{diffeo}
\includegraphics[scale=0.6]{sources/diffeo}
\end{figure}
\end{example}
\begin{question}
האם קיימים מבנים שאינם דיפאומורפיים על אותה יריעה טופולוגית?
עבור
$\dim M \leq 3$
התשובה נכונה.
עבור
$\dim M = 1$
נסו
\textbf{כתרגיל}.
עבור
$\dim M \geq 4$
הדבר תלוי ביריעה.
\end{question}
\begin{example}
בטבלה
\ref{sphere_diffeo}
מספר המבנים הדיפאומורפיים של הספירה.
עבור
$n=4$
זאת בעייה פתוחה הנקראת
\textenglish{smooth Poincaré conjecture}.
\begin{figure}[h!]
\caption{מבנים דיפאומורפיים של הספירה.}
\label{sphere_diffeo}
\centering
\begin{tabular}{|c|c|}
\hline $n$ & מספר מבנים חלקים שונים של $S^n$ עד כדי דיפאומורפיזם \\
\hline $1$ & $1$ \\
\hline $2$ & $1$ \\
\hline $3$ & $1$ \\
\hline $4$ & ? \\
\hline $5$ & $1$ \\
\hline $6$ & $1$ \\
\hline $7$ & $28$ \\
\hline $8$ & $2$ \\
\hline $9$ & $8$ \\
\hline $10$ & $6$ \\
\hline $11$ & $992$ \\
\hline $12$ & $1$ \\
\hline
\end{tabular}
\end{figure}
\end{example}
\begin{example}
ב־%
$\RR^n$
יש מבנה דיפאומורפי אחד לכל
$n \neq 4$,
ואינסוף עבור
$n=4$.
\end{example}
\section{תתי־יריעות}
\begin{definition}
תהי
$M$
יריעה טופולוגית
ממימד
$m$
ותהי
$N \subseteq M$
עם הטופולוגיה המושרית.
$N$
תיקרא
\stress{תת־יריעה טופולוגית}
אם לכל
$x \in N$
קיימת מפה
$\prs{\Uu,\phi}$
של
$M$
עם
$x \in \Uu$
וגם
$\phi^{-1}\prs{\RR^n} = \Uu \cap N$.
(כאן
$\RR^n = \set{\prs{x_1, \ldots, x_n, 0, \ldots, 0}} \subseteq \RR^m$)
\end{definition}
\begin{examples}
\begin{itemize}
\item $W \subseteq M$
קבוצה פתוחה היא תת־יריעה חלקה.
\item אם
$F \colon \RR^m \to \RR^n$
אז הגרף של
$F$,
\[\gr\prs{F} = \set{\prs{x,F(x)}}{x \in \RR^m} \subseteq \RR^m \times \RR^n\]
היא תת־יריעה.
אם
$F$
רציפה זאת רק תת־יריעה טופולוגית, אם
$F$
חלקה זאת תת־יריעה חלקה, וכו'.
תהי
$\Uu \subseteq \RR^m$
פתוחה. אז
$\Uu \times \RR^n$
פתוחה במרחב
$\RR^m \times \RR^n$.
נגדיר
\begin{align*}
\phi_{\Uu} \colon \Uu \times \RR^n &\to \Uu \times \RR^n \subseteq \RR^m \times \Rr^n \\
\prs{x,y} &\mapsto \prs{x,y-F(x)}
\end{align*}
ואז
$\gr\prs{F} \cap \prs{\Uu \cap \RR^n} \stackrel{\sim}{\longleftrightarrow} \Uu \times \set{0}$
המפה שנדרשת בהגדרה.
אם
$F$
רציפה,
$\phi_{\Uu}$
הומאומורפיזם ולכן זאת תת־יריעה טופולוגית.
אם
$F$
חלקה,
$\phi_{\Uu}$
חלקה, כלומר מתואמת עם האטלס הסטנדרטי, וזאת תת־יריעה חלקה.
\end{itemize}
\end{examples}
\begin{exercise}
גרף של
$\abs{x}$
הוא תת־יריעה טופולוגית אך לא חלקה.
\end{exercise}
\begin{exercise}\label{submanifold}
תת־יריעה טופולוגית היא יריעה טופולוגית.
\end{exercise}
\begin{proposition}
תהי
$N^n \subseteq M^m$
(כלומר
$N$
ממימד
$n$,
$M$
ממימד
$m$)
תת־יריעה חלקה. אזי
$M$
משרה על
$N$
מבנה של יריעה חלקה.
\end{proposition}
\begin{proof}
מתרגיל
\ref{submanifold}
תת־יריעה היא יריעה טופולוגית. נשאר להראות כי
$M$
משרה מבנה חלק.\\
ניקח
$\set{\prs{\Uu_x,\phi_x}}_{x \in N}$
מפות מההגדרה של תת־יריעה חלקה ונגדיר
\[\mcal{A} = \set{\prs{\Uu_x \cap N, \rest{\phi_x}{\Uu_x \cap N}}}_{x \in N}\]
אטלס שמכסה את
$N$.
נשאר להראות כי המפות מתואמות.
$\prs{\Uu_x, \phi_x}$
ו־%
$\prs{\Uu_y, \phi_y}$
מתואמות, לכן פונקציות המעבר
\[\phi_y \circ \phi_x^{-1} \colon \phi_x\prs{\Uu_x \cap \Uu_y} \to \phi_y\prs{\Uu_x \cap \Uu_y}\]
חלקות.
לכן הצמצום באטלס
$\mcal{A}$,
שהינו
\[\rest{\phi_y \circ \phi_x^{-1}}{\RR^n \cap \phi_x\prs{\Uu_x \cap \Uu_y}} \colon \RR^n \cap \phi_x\prs{\Uu_x \cap \Uu_y} \to \RR^n \cap \phi_y\prs{\Uu_x \cap \Uu_y}\]
גם הוא חלק.
\end{proof}
\begin{example}
נתונה מערכת משוואות
\[\fcases{F_1\prs{x_1, \ldots, x_m} = 0 \\
\vdots \\
F_r\prs{x_1, \ldots, x_m} = 0}\]
כאשר
$F_i$
חלקות. נניח שבנקודה
$\vec{x}$
כך ש־%
$\vec{F}\prs{\vec{x}} = 0$
מתקיים
$\rank\prs{\frac{\del \vec{F}}{\del \vec{x}}} = r$
(הדרגה מקסימלית).
ממשפט הפונקציה הסתומה ניתן לבודד
$m-r$
קואורדינטות
$\vec{x}''$
כאשר
$\vec{x}'$
שאר הקואורדינטות.
נסמן
$\vec{x} = \prs{\vec{x}'',\vec{x}'}$
בה"כ ואז לוקלית ליד
$\vec{x}$
קבוצת הפתרונות
$\vec{F}\prs{\vec{x}} = \vec{0}$
נראית כמו גרף של פונקצייה
\begin{align*}
G \colon \RR^{m-r} &\to \RR^n \\
\text{.} \phantom{G \colon \RR^{m}} x'' &\mapsto G\prs{x''} \subseteq \RR^r
\end{align*}
\end{example}
\begin{example}
עבור
$F\prs{x_1, x_2} = x_1^2 + x_2^1 - 1$,
מתקיים
$\frac{\del F}{\del x} = \prs{2x_1, 2x_2}$.
כאשר
$x_1, x_2 \neq 0$
ניתן להציג את
$x_1$
כפונקציה של
$x_2$
או להיפך.
לוקלית, ליד
$x$,
$M = \set{x}{F(x) = 0}$
היא יריעה חלקה.
אם תנאי של משפט הפונקציה הסתומנה מתקיים לכל
$x \in M$,
אז
$M$
יריעה חלקה (גלובלית).
\end{example}
\begin{exercise}
עבור תת־יריעה כמו בדוגמה, מתקיים
$\dim M = m-r$.
\end{exercise}
\begin{theorem}
נניח ש־%
$N \subseteq \RR^m$
תת־קבוצה כך שלכל
$x \in N$
קיימת סביבה
$\Uu \subseteq \RR^m$
ו־%
$r = m-n$
פונקציות חלקות
$F_i \colon \Uu \to \RR^m$
(עם
$r$
קבוע עבור
$N$)
כך שמתקיימים התנאים הבאים.
\begin{align*}
N \cap \Uu = \set{\vec{x} \in \Uu}{\vec{f}\prs{\vec{x}} = \vec{0}} \\
\rank\prs{\frac{\del \vec{F}}{\del\vec{x}}} = r
\end{align*}
אז
$N \subseteq \RR^m$
תת־יריעה ממימד
$n = m-r$.
\end{theorem}
\begin{remark}
הכיוון ההפוך איננו בהכרח נכון.
אם
$\set{\vec{x} \in \RR^n}{\vec{F}\prs{\vec{x}} = 0}$
תת־יריעה של
$\RR^m$,
ייתכן ש־%
$\vec{F}$
איננה מקיימת את תנאי משפט הפונקציה הסתומה.
המשפט הינו תנאי מספיק שאינו הכרחי.
\end{remark}
\begin{examples}
\begin{itemize}
\item $S^n$
קבוצת הפתרונות של
\[\text{.} F\prs{x_1, \ldots, x_{n+1}} = \sum_{i=1}^{n+1} x_i^2 - 1\]
מתקיים
$\prs{\frac{\del F}{\del x_i}} = \prs{2x_1, 2x_2, \ldots, 2x_{n+1}}$
ולכן
לכל
$\vec{x} \neq \vec{0}$
הדרגה היא
$1$.
$\vec{0}$
לא ביריעה, לכן זאת יריעה
$n$%
־מימדית.
\item (\textbf{תרגיל})
תהי
$B\prs{x}$
תבנית ריבועית שאיננה מנוונת.
$N = \set{x}{B(x) = 1}$
תת־יריעה.
$\set{x}{B(x) = 0}$
איננה בהכרח תת־יריעה, לדוגמה עבור
$x_1^2 + x_2^2 - x_3^2$
מתקבל חרוט דו־צדדי שאיננו תת־יריעה.
\item $\mrm{SL}\prs{n,\RR} \subseteq M_{n\times n}\prs{\RR}$
תת־יריעה כאשר
$M_{n\times n}\prs{\RR}$
מזוהה עם
$\RR^{n^2}$
עם המבנה החלק הסנדרטי.
\begin{proof}
תהי
\begin{align*}
F \colon M_{n\times n} &\to \RR \\
A &\to \det A
\end{align*}
ואז
$\mrm{SL}_n = \set{A}{FA = 1}$.
תהיינה
$\prs{x_{i,j}}_{i,j=1}^n$
קואורדינטות של
$M_{n\times n}$.
צריך להוכיח שלכל
$\vec{x} \in \mrm{SL}_n$
מתקיים
$\prs{\frac{\del F}{\del x_{i,j}}} \neq \vec{0}$,
כלומר שקיים
$x_{i,j}$
כך ש־%
$\frac{\del F}{\del x_{i,j}} \neq 0$.
נחשב.
\begin{align*}
\frac{\del F}{\del x_{i,j}} \prs{A} &= \rest{\frac{\diff}{\diff t}}{t=0} \prs{\det \prs{A + t\cdot T_{i,j}}} \\ &= \rest{\frac{\diff}{\diff t}}{t=0}\prs{\det A \cdot \det\prs{\1 + tA^{-1}T_{i,j}}} \\
&= \det A \cdot \frac{\diff}{\diff t}\prs{\det \prs{\1 + tA^{-1} T_{i,j}}} \\
&= \det A \cdot \tr \prs{A^{-1} T_{i,j}}
\end{align*}
כאשר
$\prs{T_{i,j}}_{k,l} = \delta_{k,i} \delta_{j,l}$
וכאשר
$\det\prs{1+\eps B} = 1 + \eps \tr B + O\prs{\eps^2}$
נובע מפיתוח לפי תמורות.
מתקיים
$A^{-1} = \prs{\frac{M_{i,j}}{\abs{A}}_{i,j}}$
מנוסחה עם מינורים.
מתקיים כי
$B T_{i,j}$
שווה למטריצה עם העמודה
$C_i\prs{B}$
ה־%
$i$
של
$B$
בעמודה ה־%
$j$,
ואפסים בשאר המקומות.
לכן
\[\prs{A^{-1}T_{i,j}}_{l,k} = \frac{M_{k,l}}{\abs{A}}\]
ולכן
קיבלנו
\[\text{.}\frac{\del F}{\del x_{i,j}}\prs{A} = \det A \cdot \frac{M_{j,i}}{\abs{A}} = M_{i,j}\]
$A$
הפיכה ולכן לא כל המינורים מתאפסים ונקבל כי
$\prs{\frac{\del F}{\del x_{i,j}}} \neq \vec{0}$.
לכן גם
$\dim \mrm{SL}\prs{n,\RR} = n^2 - 1$.
\end{proof}
\item \begin{exercise}
$O\prs{n} \leq M_{n\times n}$
תת יריעה.
\end{exercise}
\item $\TT^n \subseteq \RR^{2n}$
כאשר
$\TT^n \cong \prs{S^1}^n$
ו־%
$\RR^{2n} \cong \prs{\RR^2}^n$,
ניתן להצגה ע"י המשוואות
$x_{2k-1}^2 + x_{2k}^2 - 1 = 0$
כאשר
$k \in \brs{n}$.
עבור
$\TT^2$,
ניתן לכתוב
\[\text{.} F\prs{x_1, x_2, x_3} = \prs{\sqrt{x_1^2 + x_2^2} - 2}^2 + x_3^2 - 1 = 0\]
\end{itemize}
\end{examples}
\begin{exercise}
$\nabla F \neq \vec{0}$
בנקודה
$x \in \TT^2$.
\end{exercise}
\begin{exercise}
ניקח $\TT^2 = \quot{\RR^2}{\ZZ^2}$
ונגדיר
\[\fcases{\phi_1(t) = \phi_1(0) + v_1 t \\
\phi_2\prs{t} = \phi_2(0) + v_2 t}\]
תנועה חופשית של חלקיק ב־%
$\TT^2$.
לאילו
$v_i$
המסלול הוא תת־יריעה של
$\TT^2$?
\end{exercise}
\begin{example}
תהי
$f \colon M \to N$
חלקה.
אזי
\[\gr\prs{f} = \set{\prs{x,f(x)}}{x \in M} \subseteq M \times N\]
תת־יריעה.
\stress{האלכסון}
הוא
$\Delta \subseteq M \times M$
הגרף של
$\1_{M}$.
הכייון ההפוך איננו נכון. אם
$\gr{f}$
תת־יריעה, לא מובטח כי
$f$
חלקה.
\end{example}
\begin{remark}
ברוב הדוגמאות, יריעות הן תתי־יריעות של
$\RR^N$.
\end{remark}
\begin{theorem}[\textenglish{Whitney}]
כל יריעה
$M^m$
ניתנת לשיכון כת־יריעה חלקה ב־%
$\RR^N$
עבור
$N$
מסוים.
\end{theorem}
\begin{remark}
המקרה האופטימלי הכללי הוא
$N=2m$.
ל־%
$2m-1$
אין תמיד שיכון כתת־יריעה חלקה.
$\RR\mrm{P}^2, K^2$
לא ניתנים לשיכון ב־%
$\RR^3$.
\end{remark}
\begin{exercise}
תהי $N \subseteq \RR^m$
תת־יריעה ויהי
$x \in N$.
$f \colon N \to \RR$
חלקה ליד
$x$
אם ורק אם ניתן להרחיב את
$f$
לפונקציה
$F \colon \Uu \subseteq \RR^m \to \RR$
חלקה, כאשר
$\Uu$
סביבה של
$x$
ב־%
$\RR^m$.
\end{exercise}
\section{נגזרות}
\begin{definition}
תהי
$M$
יריעה חלקה.
\stress{עקומה}
$\gamma$
היא העתקה
$\gamma \colon \prs{a,b} \to M$
חלקה (ביחס למבנה החלק של
$M$
והסטנדרטי ב־%
$\RR$).
\end{definition}
\begin{question}
מהי
$\dot{\gamma}(x)$,
"וקטור המהירות של חלקיק הנע לאורך
$\gamma$"?
אם
$M \subseteq \RR^n$
\textbf{עם שיכון נתון}
נוכל להסתכל על
$\vec{\gamma} = \pmat{\gamma_1\prs{t} \\ \vdots \\ \gamma_n\prs{t}}$
ואז לגזור
$\dot{\gamma}(x) = \pmat{\dot{\gamma}_1\prs{t} \\ \vdots \\ \dot{\gamma}_n\prs{t}}$
וקטור שמשיק ל־%
$M$
בנקודה
$\gamma\prs{x}$.
\end{question}
\begin{remark}
תהי
$M$
יריעה חלקה.
היינו רוצים להגדיר מרחב משיק בנקודה
$p$
להיות אוסף וקטורי המהירות של מסילות העוברות דרך
$p$.
אין לנו דרך טובה להגדיר וקטורי מהירות באופן כללי, לכן צריך לעקוף זאת.
\end{remark}
\begin{definition}
תהיינה
$\gamma_i \colon \prs{-\eps, \eps} \to M$
מסילות עבור
$i \in [2]$
וכאשר
$\gamma_i(0) = p$.
אזי
$\gamma_1 \sim \gamma_2$
אם קיימת מפה
$\prs{\Uu, \phi}$
סביב
$p$
כך שלהצגות המקומיות
$\hat{\gamma}_i = \phi \circ \gamma_i$
אותו וקטור מהירות
$\dot{\hat{\gamma}}_1(0) = \dot{\hat{\gamma}}_2(0)$.
\end{definition}
\begin{definition}
\stress{וקטור משיק בנקודה
$p$}
הוא מחלקת שקילות של עקומות
$\bar{\gamma}$
עם
$\gamma(0) = p$.
\end{definition}
\begin{exercise}
הראו כי שקילות עקומות בהגדרה הנ"ל היא אכן יחס שקילות.
\end{exercise}
\begin{remark}
יחס השקילות
$\sim$
אינו תלוי בבחירת המפה סביב
$p$.
נסתכל על שתי מפות
$\phi, \psi$
ואז
\[\underset{\hat{\gamma}_i}{\underbrace{\phi \circ \gamma_i}} = \prs{\phi \circ \psi^{-1}} \circ \underset{\hat{\hat{\gamma}}_i}{\underbrace{\psi \circ \gamma_i}}\]
ומכלל השרשרת
\[\dot{\prs{\hat{\gamma}_i}}(0) = \rest{D\prs{\phi \circ \psi^{-1}}}{\phi\prs{p}} \cdot \dot{\prs{\hat{\hat{\gamma}}_i}}(0)\] %TODO fix, dot over pars over hats over gamma
ואז
%TODO dot over two hats equal iff dot only one hat (i=1,2)
\end{remark}
\begin{definition}
נגדיר
\[T_p M = \quot{\set{\text{smooth paths $\gamma(0)=p$}}}{\sim}\]
\stress{המרחב המשיק בנקודה
$p \in M$}.
\end{definition}

\begin{remark}
$\phi \colon M \to \RR^m$
מגדירה העתקה

\begin{align*}
D\phi_p \colon T_p M &\to \RR^m \\
\brs{\gamma\prs{t}} &\to \dot{\prs{\phi \circ \gamma}}(0)
\end{align*}

\end{remark}
\begin{exercise}
\begin{enumerate}
\item \[D \phi_p \colon T_p M \to \RR^m\]
חח"ע ועל.
\item בהינתן שתי מפות
$\prs{\Uu,\phi}$, $\prs{\Vv,\psi}$
סביב
$p$,
הדיאגרמה הבאה קומוטטיבית.
\[
\begin{tikzcd}
& T_p M \arrow[dl, swap, "D \phi_p"] \arrow[dr, "D \psi_p"] \\
\RR^m \arrow[rr, swap, dashed, "\rest{D\prs{\psi \circ \phi^{-1}}}{\phi(p)}"] & & \RR^m
\end{tikzcd}
\]
\item
ל־%
$T_p M$
יש מבנה לינארי טבעי, ע"י משיכת המבנה הלינארי מ־%
$\RR^m$.
נגדיר
\begin{align*}
\fcases{ \sigma + \eta \ceq D\phi_p^{-1} \prs{D \phi_p\prs{\sigma} + D\phi_p\prs{\eta}} \\
c \cdot \sigma \ceq D\phi_p^{-1} \prs{c \cdot D\phi_p\prs{\sigma}}
}
\end{align*}
ונשאיר כתרגיל בדיקה כי הדבר אינו תלוי בבחירת המפה
$\phi$.
(זה נובע מכך שהעתקות המעבר הינן לינאריות)
\end{enumerate}
\end{exercise}
\begin{example}
תהי
$\Uu \subseteq \RR^n$
פתוחה וניקח את המפה
$\prs{\Uu, \1_{\Uu}}$.
אז יש איזומורפיזם
\begin{align*}
D\1_{\Uu} \colon T_p \Uu &\to \RR^n \\
\text{.} \phantom{D\1_{\Uu} \colon T}\brs{\gamma} &\to \dot{\gamma}(0)
\end{align*}
\end{example}
\begin{example}
תהי יריעה
$M^{n-1} \subseteq \RR^n$
מוגדרת על ידי
$\set{\vec{x}}{F\prs{x_1, \ldots, x_n} = 0}$
עם
$\nabla F \neq \vec{0}$.
כעת
$T_p M \subseteq T_p \RR^n \cong \RR^n$.
מהו
$T_p M$
אחרי הזיהוי של
$T_p\RR^n$
עם
$\RR^n$?
ניקח
$\gamma \colon \prs{-\eps, \eps} \to M$
עם
$\gamma(0) = p$.
נכתוב
$\gamma\prs{t} = \pmat{\gamma_1(t) \\ \vdots \\ \gamma_n(t)} \in \RR^n$.
כעת
$\gamma(t) \in M$
גורר כי לכל
$t \in \prs{-\eps, \eps}$
מתקיים
$F \circ \gamma\prs{t} = 0$.
נחשב בעזרת כלל השרשרת.
\begin{align*}
\rest{\frac{\diff}{\diff t}\prs{F \circ \gamma\prs{t}}}{t=0} &= \rest{\frac{\diff}{\diff t}}{t=0} F\prs{\gamma(t)} \\&=
\rest{\sum \frac{\del F}{\del x_i}}{\gamma(0)} \dot{\gamma}_i(0) \\&=
\rest{\nabla F}{\gamma(0)} \cdot \dot{\gamma}(0)
\end{align*}
לכן אם
$\sigma \in T_p M$
אז
$\sigma = D \phi_p\prs{\sigma} \perp \nabla F (p)$.
אם נזהה את
$T_pM$
עם
$D \phi_p \prs{T_p M}$
אזי
$T_p M \subseteq \prs{\nabla F}^{\perp}$.
אלו שני מרחבים וקטוריים ממימד
$n-1$,
לכן יש שיוויון
$T_p M  = \prs{\nabla F}^{\perp}$.
\end{example}
\begin{example}
$S^1 \subseteq \RR^2$
מוגדר ע"י
$F\prs{x_1, x_2} = x_1^2 + x_2^2 - 1 =0$
עם
$\nabla F = \prs{2x_1, 2x_2}$.
$T_p M + p$
הינו ישר המאונך ל־%
$\nabla F(p)$,
וזהו בדיוק ישר המשיק ל־%
$S^1$
בנקודה
$p$.
\end{example}
\begin{definition}
\stress{אגד משיק}\textenglish{tangent bundle}
הינו
\[\text{.}TM = \coprod_{x \in M} \prs{x,T_x M}\]
איבר ב־%
$TM$
הינו זוג
$\prs{x, \sigma}$
כאשר
$x \in M$
ו־%
$\sigma \in T_xM$.
\end{definition}
\begin{example}
אגד משיק למעגל הוא
$TM = S^1 \times \RR$,
כלומר גליל.
\end{example}
\begin{remark}
על
$TM$
טופולוגיה הניתנת באופן מקומי על ידי אטלסים.
\end{remark}
\begin{remark}
קיים זיהוי קנוני בין
$T_p \Uu, T_p M$
עבור
$\Uu \subseteq M$
פתוחה.
\end{remark}
\begin{example}
תהי
$\Uu \subseteq \RR^n$
פתוחה. אז
\[T\Uu = \set{\prs{x,\sigma}}{x \in \Uu \,\wedge\, \sigma \in T_x \Uu \cong \RR^n} \cong \Uu \cong \RR^n\]
או וקטורים עם נקודת התחלה ב־%
$\Uu$
וחץ שהוא וקטור ב־%
$\RR^n$.
\end{example}

\begin{definition}
עבור
$M$
יריעה כללית, יהי
$\Aa = \set{\prs{\Uu_{\alpha}, \phi_{\alpha}}}_{\alpha \in I}$
אטלס חלק על
$M$.
נגדיר
\[\bar{\Uu}_{\alpha} \ceq \coprod_{x \in \Uu_{\alpha}} T_x \Uu_{\alpha}\]
וגם
\begin{align*}
\bar{\phi}_{\alpha} \colon \bar{\Uu}_{\alpha} &\to T\brs{\phi_{\alpha}\prs{\Uu_{\alpha}}} \cong \phi_{\alpha}\prs{\Uu_{\alpha}}\times \RR^n \subseteq \RR^{2n} \\
\prs{x, \sigma} &\to \prs{\phi_{\alpha}(x), D\phi_{\alpha}(x)(\sigma)} \in T_{\phi_{\alpha}(x)}\phi_{\alpha}\prs{\Uu_{\alpha}}
\end{align*}
\end{definition}

\begin{exercise}
\begin{enumerate}
\item \[TM = \bigcup_{\alpha \in I} \bar{\Uu}_{\alpha}\]
\item \[\bar{\phi}_{\alpha} \colon \bar{\Uu}_{\alpha} \to \phi_{\alpha}\prs{\Uu_{\alpha}}\times \RR^n\]
חח"ע ועל.
\end{enumerate}
\end{exercise}
\begin{definition}
תת־קבוצה
$X \subseteq TM$
נקראת פתוחה אם ורק אם לכל
$\alpha$,
$\bar{\phi}_{\alpha}\prs{X \cap \bar{\Uu}_{\alpha}}$
פתוחה ב־%
$\RR^{2n}$.
\end{definition}
\begin{exercise}
\begin{enumerate}
הראו בשלבים הבאים כי יש מבנה חלק על
$TM$.
\item ההגדרה מגדירה טופולוגיה על
$TM$.
\item $TM$
יריעה טופולוגית.
\item $\set{\prs{\bar{\Uu}_{\alpha}, \bar{\phi}}}_{\alpha \in I}$
אטלס מתואם.
\end{enumerate}
\end{exercise}
\begin{remark}
$TS^{1} \cong S^1 \times \RR$,
אבל
$TS^2 \not\cong S^2 \times \RR^2$.
\end{remark}
\backmatter
\end{document}

